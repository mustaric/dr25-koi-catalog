\label{s:mcmc}

Each KOI, whether from a previous catalog or new to the DR25 catalog, was fit with a transit model in a consistent manner. The model fitting was performed in a similar to that described in \S5 of \citet{Rowe2015cat}. The model fits start by detrending the DR25 Q1--Q17 PDC photometry from MAST\footnote{\url{https://archive.stsci.edu/kepler/}} using a polynomial filter as described in \S4 of \citet{Rowe2014}. A transit model based on \citet{Mandel2002} is fit to the photometry using a Levenberg-Marquardt routine \citep{More1980} assuming circular orbits and using fixed quadratic limb darkening coefficients \citep{Claret2011} calculated using the DR25 stellar parameters \citep{Mathur2017ApJS}. TTVs are included in the model fit when necessary; the calculation of TTVs follows the procedure described in \S4.2 of \citet{Rowe2014}. The 296 KOIs with TTVs and the TTV measurements of each transit are listed in Table~\ref{t:ttv}. The uncertainties for the fitted parameters were calculated using a Markov-chain Monte Carlo (MCMC) method \citep{Ford2005AJ} with a single chain with a length of 2$\times 10^{5}$ calculated for each fit. In order to calculate the posterior distribution the first 20\% of each chain was discarded. The transit model fit parameters were combined with the DR25 stellar parameters and associated errors \citep{Mathur2017ApJS} in order to produce the reported planetary parameters and associated errors. The MCMC chains are all available at the Exoplanet Archive and are documented in \citet{Hoffman2017}. 

The listed planet parameters come from the least-squares (LS) model fits and the associated errors from the MCMC calculations. Note that not all KOIs could be successfully modelled, resulting in three different fit types: LS+MCMC, LS, and none. In the case of LS+MCMC the KOIs were fully modelled with both a least-squares model fit and the MCMC calculations were completed to provide associated errors. In the cases where the MCMC calculations did not converge, but there is a model fit, the least-square parameters are available without uncertainties (LS fit type). In the final case, where a KOI could not be modelled (e.g., cases where the transit event was not found in the detrending used by the MCMC fits) only the period, epoch, and duration of the federated TCE are reported and the fit type is listed as none.

\begin{deluxetable}{crrr}
\tabletypesize{\scriptsize}
\tablecaption{TTV Measurements of KOIs}
\tablewidth{0pt}
\tablehead{
\colhead{$n$}  & \colhead{$t_n$} & \colhead{$TTV_n$}  & \colhead{$TTV_{n\sigma}$} \\
\colhead{}     & \colhead{BJD-2454900.0}      & \colhead{days}     & \colhead{days}
}
\startdata
KOI-6.01 & & & \\
1 & 54.6961006 & 0.0774247 &  0.0147653 \\ 
2 & 56.0302021 & -0.0029102 &  0.0187065 \\ 
3 & 57.3643036 & -0.0734907 &  0.0190672 \\ 
4 & 58.6984051 & 0.0119630 &  0.0176231 \\ 
\nodata & \nodata & \nodata & \nodata\\
KOI-8.01 & & & \\
1 & 54.7046603 & -0.0001052 &  0.0101507 \\ 
2 & 55.8648130 & -0.0103412 &  0.0084821 \\ 
3 & 57.0249656 & 0.0047752 &  0.0071993 \\ 
\nodata & \nodata & \nodata & \nodata\\
KOI-8151.01 & & & \\
1 & 324.6953389 & 0.1093384 &  0.0025765 \\ 
2 & 756.2139285 & -0.3478332 &  0.0015206 \\ 
3 & 1187.7325181 & 0.0110542 &  0.0016480 \\
\nodata & \nodata & \nodata & \nodata\\
\enddata
\tablecomments{Column 1, $n$, is the transit number. Column 2, $t_n$, is the transit time in Barycentric Julian Date minus the offset 2454900.0. Column 3, $TTV_n$, is the observed - calculated (O-C) transit time. Column 4, $TTV_{n\sigma}$,  is the $1\sigma$ uncertainty in the O-C transit time.
Table \ref{t:ttv} is published in its entirety in the electronic edition of the {\it Astrophysical Journal Supplement}.  A portion is shown here for guidance regarding its form and content.}
\label{t:ttv}
\end{deluxetable}
