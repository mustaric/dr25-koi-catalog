\section{Considerations and Products to Aide in DR25 Catalog Occurrence Rate Calculations}
\label{s:occurates}
The DR25 candidate catalog was designed with the goal of providing a well characterized catalog of planetary candidates so that it can be used to do occurrence rate calculations.  For those smallest planets at the longest periods, our vetting is especially prone to miss transits and confuse other signals as transits, and this must be accounted for when doing occurrence rates.  However, the completeness and reliability presented in this paper is simply the last two pieces of a much larger puzzle that must be put together in order to perform occurrence rates with this catalog.  In this section we endeavor to make users aware of the products that are available to fully understand the biases in this candidate catalog.  All of the products discussed her are available at the NASA exoplanet archive. 

\subsection{Pipeline Detection Efficiency}
Any measure of the catalog completeness must include the completeness of the Robovetter and the \Kepler{} Pipeline.  The Pipeline's detection efficiency has been explored in two ways: using pixel-level transit injection and using flux-level transit injection.  Pixel level transit injection gives an average response of the transit-search over all the stars that were searched. A full description of the signals that were injected and recovered can be found in \citet{Christiansen2017}.  To understand the recoverability of transiting planets on individual targets, 600,000 (or 2,000) transiting signals were injected on individual stars to see which ones would be recovered by TPS.  This led to an understanding of how to identify those stars that deviate from the average detection efficiency measured from the pixel-level injection.  The pixel-level measurements have the advantage of following transit signals through all processing steps of the \Kepler{} Pipeline, and the recovered signals can be further classified with the Robovetter, as demonstrated in \S\ref{s:candr}.  However, since the pixel-level injection only includes one injection per target, it does not elucidate individual-target variations in pipeline completeness due to differences in stellar properties or astrophysical variability. The flux-level injections revealed  that  there  are  significant target-to-target variations in the detection efficiency. The flux-level injections and the resulting detection efficiency is available for the sample of stars that were part of this study. For more information on the flux-level injection study see \citet{Burke2017c}. All products associated with the flux-level and pixel-level injections can be found at NexScI\footnote{\url{https://exoplanetarchive.ipac.caltech.edu/docs/KeplerSimulated.html}}.

[IT WOULD BE NICE TO COMBINE THE PIPELINE AND VETTING COMPLETENESS AND SHOW IT HERE -- JESSIE??]

\subsection{Astrophysical Reliability}
We have described the reliability of the DR25 candidates with regard to the possibility that the observed period and epoch is actually caused by stellar or instrumental noise. See \S\ref{s:candr} for how this reliability varies with various measured parameters.  Even if the observed signal is real, other astrophysical events can mimic a transit \citep{Morton2016}. Some of these other astrophysical events are removed by carefully vetting the KOI with \Kepler{} data alone.  Specifically, the Robovetter looks for significant secondary eclipses to rule out eclipsing binaries, and for a significant offset in the location of the in- and out-of-transit to rule out background eclipsing binaries.  Using the pixel-level transit injection, we inject signals that mimic eclipsing binaries and signals that mimic background eclipsing binaries. Those that were recovered by the \Kepler{} Pipeline can be used to measure how effective the Robovetter was a removing this type of false positive. Additional information on the effectiveness can be gathered by using the Certified False Positive table at NExScI\footnote{\url{https://exoplanetarchive.ipac.caltech.edu/cgi-bin/TblView/nph-tblView?app=ExoTbls\&config=fpwg}} which uses follow-up information to determine which KOIs are truly false positives. The Robovetter effectiveness can be combined with the occurrence rate of the different types of astrophysical events to determine the astrophysical reliability of planet candidates. The results of the VESPA code \citep[][]{Morton2016} for the KOIs are available in the False Positive Probabilities Table \footnote{\url{https://exoplanetarchive.ipac.caltech.edu/cgi-bin/TblView/nph-tblView?app=ExoTbls\&config=koifpp}}.

For those doing occurrence rates, another issue to consider is whether the measured size of the planet is correct. The stellar radii and temperatures provided by \citet{Mathur2017ApJS} collate the best information available at the time about the \Kepler{} stars, but has typical errors of 27 percent for the stellar radii. Results from Gaia \citep{Gaia2016} are expected to fix most of shortcomings of this catalog. Also, as studied by \citep{Ciardi2015} and again by \citep{Furlan2017} $\sim$30 per cent of KOI host stars are part of a bound or line-of-sight binary within one Kepler pixel (4 arc seconds) of the primary star. While the \Kepler{} Pipeline does account for stray light in the aperture from stars listed in the Kepler Input Catalog \citep{Brown2011}, these binaries are only known because of recent high resolution imaging. As a result the observed transits are diluted by unaccounted stars. In these cases the planet size is actually larger to account for the measured transit depth. For occurrence rates this also affects the stars that have no planets because it means the search did not extend to planet radii that are as small as originally thought.  For this reason, any correction based on light from observed binaries needs to be applied across all searched stars, not just the planet hosts.


%The reliability of individual objects against can be better understood by consulting the 

%Do I need to mention APP and FPWG. I don't quite see how to use them for getting a handle on reliability. 
%Several tools are also available to understand whether an individual target is likely caused by an astrophysical source. the reliability of the The probabilities provided in the APP table measure how likely it is that a star’s location matches the location of the transit signal – they do not measure the probability that the transit signal is consistent with a planet orbiting that star.