\section{Using the Catalog for Occurrence Rates}

The DR25 candidate catalog was designed with the goal of providing a well characterized catalog of candidates so that it can be used to do occurrence rate calculations.  For those smallest planets at the longest periods, our catalog is especially prone to miss transits and confuse other signals as transits.  Therefore one must understand the completeness of the entire process as well as the reliability of those signals in the catalog.  The completeness and reliability presented in this catalog is simply the last two pieces of a much larger puzzle that must be put together in order to perform occurrence rates with this catalog.  In this section we endeavor to make users aware of the products that are available to do a full occurrence rate calculation with this DR25 catalog as the list of planet candidates used in that calculation. All of the products discussed her are available at the NASA exoplanet archive. 

Any measure of the catalog completeness must include the completeness of the Robovetter and the \Kepler\ pipeline.  The pipeline's detection efficiency has been explored in two ways: using pixel-level transit injection and using flux-level transit injection.  Pixel level transit injection gives an average response of the transit-search over all the stars that were searched. A full description of the signals that were injected and recovered can be found in \citet{KSCI19111}.  To understand the recoverability of transiting planets on individual targets, 600,000 (or 2,000) transiting signals were injected on individual stars to see which ones would be rcovered by TPS.  This led to an understanding of how to identify those stars that deviate from the average detection efficiency measured from the pixel-level injection.  The pixel-level measurements have the advantage of following transit signals through all processing steps of the \Kepler{} pipeline, and the recovered signals can be further classified with the Robovetter, as demonstrated in \S\ref{s:candr}.  However, since the pixel-level injection only include one injection per target, it does not elucidate individual-target variations in pipeline completeness due to differences in stellar properties or astrophysical variability. the flux-level injections revealed  that  there  are  significant target-to-target variations in the detection efficiency. The flux-level injections and the resulting detection efficiency is available for the sample of stars that were part of this study. For more information on the flux-level injection study see \citet{KSCI19109}. 


We have described the reliability of the DR25 candidates with regard to the possibility that the observed period and epoch is actually caused by stellar or instrumental noise. See \S\ref{s:candr} for how this reliability varies with various measured parameters.  Even if the observed signal is real, other astrophysical events can mimic a transit \citep{Morton2016}. Some of these other astrophysical events are removed by carefully vetting the \Kepler{} data alone.  The Robovetter looks for significant secondary eclipses to rule out eclipsing binaries, and for a significant offset in the location of the in- and out-of-transit to rule out background eclipsing binaries.  Using the pixel-level transit injection, we injected signals that mimic eclipsing binaries and signals that mimic a background eclipsing binaries.  Those that were turned into TCEs can be used to measure how effective the Robovetter was a removing this type of false positive. This effectiveness can be used to improve likelihood measurements of the non-transiting exoplanet scenarios as done by \citet{Morton2016}.




%The reliability of individual objects against can be better understood by consulting the 

%Do I need to mention APP and FPWG. I don't quite see how to use them for getting a handle on reliability. 
%Several tools are also available to understand whether an individual target is likely caused by an astrophysical source. the reliability of the The probabilities provided in the APP table measure how likely it is that a star’s location matches the location of the transit signal – they do not measure the probability that the transit signal is consistent with a planet orbiting that star.