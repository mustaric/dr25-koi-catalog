\def \nkois {8054}
\def \ncand {4034}
\def \newkois {738}
\def \newcand {210}
\def \completeness {85.21}
\def \reliability {97.14}
\def \effectiveness {99.6}
%For above numbers see script-playPublicData

\subsection{Summary of the Exoplanet Catalog}

The final DR25 catalog, available at NExScI contains all TCEs that pass the not-transit like tests (\S\ref{nottransitlikesec}). 


Some overall statistics of the DR25 KOI catalog are as follows:
\begin{itemize}
    \item \nkois{}~KOIs
    \item \ncand{}~PCs
    \item \newkois{}~new KOIs
    \item \newcand{}~PCs on new KOIs
    \item \completeness{} per cent of \injtce{s} are PCs
    \item \effectiveness{} per cent of \invtce{s} and \scrtce{s} are FPs
\end{itemize}

A summary of the planet radii and period of the planet candidates available in this catalog is shown in Figure~\ref{f:catalogPlot}. A clear excess of candidates exists with periods near to 370\,d;  with a score cut of $0.7$, this excess disappears. While the disposition score provides an easy way to make an additional cut on the PC population at long periods, when discussing the catalog PCs below we are using the pure dispositions of the Robovetter unless otherwise stated. The deficit of planets with radii just below 2.0\,R$_{\oplus}$ is consistent with the study of \citet{Fulton2017} where they report a natural gap in the abundance of planets between super-Earths and sub-Neptunes by applying precise stellar parameters to a subset of the \kepler\ transiting candidates. The new KOIs with a disposition of PC are found at all periods, but only 10 have MES$\geq 10 $. 

\begin{figure*}[ht]
    \centering
    \includegraphics[width=1.1\linewidth]{fig-radiusPeriodScore-hist.png}
    \caption{DR25 planet candidates plotted as planet radius against Period with the color representing the disposition score. Those plotted in orange and yellow are those whose metrics lie near to the lest confident PCs.  The period and planet radii distributions are plotted on the top and on the left, respectively, in blue. The red line shows the distributions if you only consider those KOIs with a score greater than 0.8. }
    \label{f:catalogPlot}
\end{figure*}

\subsection{Compare Dispositions to Other Catalogs}
We compare the catalog to two sets of \Kepler\ exoplanets: the confirmed exoplanets and the certified false positives.  In both of these cases, additional observations and careful vetting are used to verify the signal as either an exoplanet or a Certified False Positive. It is worth comparing the Robovetter to these catalogs as a sanity check.  

We use the confirmed exoplanet list from NExScI\footnote{https://exoplanetarchive.ipac.caltech.edu/cgi-bin/TblView/nph-tblView?app=ExoTbls\&config=planets} on 2017-05-24.  2279 confirmed planets are in the DR25 KOI catalog.  The DR25 Robovetter fails 44 confirmed planets, less than 2 per cent. Half of these FPs are not transit-like fails, 16 are stellar eclipse fails, six are centroid offsets and one is an ephemeris match. Twelve fail due to the LPP metric, all have periods less than 50 days.  The LPP metric threshold was set to improve the reliability of the long period KOIs, an act which sacrificed some of the short period KOIs.  The reason the Robovetter failed each confirmed planet is given in the ``koi\_comment" column at NExScI (see \S\ref{s:minorflags}). 

For the vast majority of these confirmed FPs, careful inspection reveals that there is no doubt that the Robovetter is incorrect. As an example, Kepler-10b \citep[][]{Batalha2011Kepler10,Fogtmann2014Kepler10}, a rocky planet in a 0.84~days orbit, was failed due to the LPP metric. This occurred because the detrending algorithm used by the \Kepler\ pipeline significantly distorts the shape of the transit, a known problem for strong, short period signals \citep{Christiansen2015} that the LPP metric does not account for.

In some cases the Robovetter may be giving the correct disposition.  Many of the confirmed planets are only statistically validated \citep{Morton2016,Rowe2014}. In these cases no additional data exists proving the existence of a planet outside of the transits observed by \Kepler. It is possible that the DR25 light curves and metrics have now revealed evidence that the periodic events are caused by noise or a binary star. For example, Kepler-367c \citep{Rowe2014}, Kepler-1507b \citep{Morton2016} and Kepler-1561b \citep{Morton2016}, (K02173.02, K3465.01 and K04169.01 respectively) were all confirmed by validation and have now failed the Robovetter because of the new ghost metric \S\ref{s:ghost}, indicating that the events are caused by a contaminating source not localized to the target star.  These validations should be revisited in the presence of these new results.

% state what the accuracy of the statistical validation.

We use the Certified False Positive table\footnote{\url{https://exoplanetarchive.ipac.caltech.edu/cgi-bin/TblView/nph-tblView?app=ExoTbls\&config=fpwg}} downloaded from NExScI on 2017-07-11. This table contains objects known to be a false positive based on all available data, including ground-based follow-up information.  The Robovetter passes 106 of the 2713 certified false positives known at the time, only 3.9 per cent.  Most of those called a PC by the Robovetter are high signal-to-noise and more than half have periods less than 5 days.  The most common reason it is a certified false positive is that there is evidence it is an eclipsing binary. In some cases, external information, like radial velocities provide a mass which determines that the KOI is actually a binary system. The other dominant reason for the discrepancy between tables is that the certified false positive table shows evidence of a significant centroid offset. In crowded fields the Centroid Robovetter (\S\ref{s:centroidrv}) will not fail observed offsets because of the potential for confusion. For the Certified False Positive table, individual cases are examined by a team of scientist who determines when there is sufficient proof that the signal is indeed caused by a background eclipsing binary.  

