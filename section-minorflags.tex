%List of the Minor Flags and their definitions
% \clearpage
\section{Minor False Positive Flag Definitions}
\label{s:minorflags}
The Robovetter produces a flag each time it gives a disposition of FP, and sometimes when it gives a disposition of PC.  Here we give a definition for each flag.   These flags are available for the KOIs through the comment column in the KOI table at the Exoplanet Archive. See the Robovetter output files\footnote{The Robovetter output files have the format kplr\_dr25\_XXX\_robovetter\_output.txt (XXX represents the data set name) and can be found in the Robovetter github repository, \url{https://github.com/nasa/kepler-robovetter}} for the flags for all the \opstce{s}, \injtce{s}, \invtce{s}, \scrtce{s}. A summary of the Robovetter metrics is given in Table~\ref{t:metrics}.


\textbf{ALL\_TRANS\_CHASES}: This flag is set when the per TCE Chases metric is above threshold. This indicates that the shapes of the individual transits are generally not reliable and the TCE is dispositioned as an FP with the not transit-like major flag set. See \S\ref{s:tcechases}.

\textbf{CENT\_CROWDED}: This flag is set as a warning that more than one potential stellar image was found in the difference image, and thus a reliable centroid measurement cannot be obtained. See \S\ref{s:centroidrv}.

\textbf{CENT\_FEW\_DIFFS}: Fewer than 3 difference images of sufficiently high SNR are available, and thus very few tests in the pipeline's centroid module are applicable to the TCE. If this flag is set in conjunction with the CENT\_RESOLVED\_OFFSET flag, it serves as a warning that the source of the transit may be on a star clearly resolved from the target. See \S\ref{s:centroidrv}.

\textbf{CENT\_FEW\_MEAS}: The PRF centroid fit used by the pipeline's centroid module does not always converge, even in high SNR difference images. This flag is set as a warning if centroid offsets are recorded for fewer than 3 high SNR difference images. See \S\ref{s:centroidrv}.

\textbf{CENT\_INVERT\_DIFF}: One or more difference images were inverted, meaning the difference image claims the star got brighter during transit. This is usually due to variability of the target star and suggests the difference image should not be trusted. When this flag is set, it is a warning that the TCE requires further scrutiny, but the TCE is not marked as an FP due to a centroid offset. See \S\ref{s:centroidrv}.

\textbf{CENT\_KIC\_POS}: This measured offset distance is relative to the star's recorded position in the Kepler Input Catalog (KIC), not the out of transit centroid. Both are useful, since the KIC position is less accurate in sparse fields, but more accurate in crowded fields. If this is the only flag set, there is no reason to believe a statistically significant centroid shift is present. See \S\ref{s:centroidrv}.

\textbf{CENT\_NOFITS}: The transit was not fit by a model in DV and thus no difference images were created for use by the pipeline's centroid module, so this flag is set as a warning that the TCE cannot be evaluated. This flag is typically set for very deep transits due to eclipsing binaries. See \S\ref{s:centroidrv}.

\textbf{CENT\_RESOLVED\_OFFSET}: The TCE has a significant centroid offset because the transit occurs on a star that is spatially resolved from the target. The TCE is marked as an FP with the centroid offset flag set unless one of the other Centroid Robovetter flags is also set, casting doubt on the measurement. See \S\ref{s:centroidrv}.

\textbf{CENT\_SATURATED}: The star is saturated, so the Robovetter's centroiding assumptions break down. This flag is set as a warning, indicating that the TCE cannot be reliably evaluated.  See \S\ref{s:centroidrv}.

\textbf{CENT\_UNCERTAIN}: The significance of the centroid offset cannot be measured to high enough precision, so this flag is set as a warning that the TCE cannot be confidently dispositioned as an FP. This is typically due to having only a very small number (i.e., 3 or 4) of offset measurements, all with low SNR. See \S\ref{s:centroidrv}.

\textbf{CENT\_UNRESOLVED\_OFFSET}: There is a statistically significant shift in the centroid during transit. This indicates the is not on the target star. Thus, the TCE is dispositioned as an FP with the centroid offset major flag set, unless another Centroid Robovetter flag is also set, casting doubt on the measurement. See \S\ref{s:centroidrv}.

\textbf{DEEP\_V\_SHAPED}: The V-shape metric is above threshold. This metric uses the fitted DV radius ratio and impact parameter to determine whether the event is likely to be caused by a stellar eclipse. When the flag is set, the TCE is dispositioned as an FP with the stellar eclipse major flag set. See \S\ref{s:shapemetric}.

\textbf{DEPTH\_ODDEVEN\_(ALT|DV)}: The TCE failed the odd-even depth test using the ALT or DV detrending. This determines whether the difference in the depths of the odd and even transits is greater than the standard deviation of the measured depths. The transit-like TCE is marked as an FP with a stellar eclipse major flag set. See \S\ref{s:oddeven}.

%\textbf{DEPTH\_ODDEVEN\_DV}: The TCE failed the odd-even depth test using the DV detrending. This determines whether the difference in the depths of the odd and even transits is greater than the standard deviation of those depths. The transit-like TCE is marked as an FP with a stellar eclipse major flag set.

\textbf{EPHEM\_MATCH}: The TCE has been identified as an FP due to an ephemeris match with a source that could plausibly induce the observed variability on the target. See \S\ref{ephemmatchsec} and Table~\ref{ephemmatchtab} for the contaminating source.

\textbf{HALO\_GHOST}: The ghost diagnostic value is too high. This diagnostic measures the transit strength for the out- and in-aperture pixels and determines if the transit is localized on the target star, or if it is due to contamination from a distant source. The TCE is an FP and the centroid offset major flag is set. See \S\ref{s:ghost}.

\textbf{HAS\_SEC\_TCE}: Another TCE on the same target with a higher planet number has the same period as the current transit-like TCE, but a significantly different epoch. This indicates that the current TCE is an eclipsing binary with the other TCE representing the secondary eclipse. If the PLANET\_OCCULT\_DV and PLANET\_OCCULT\_ALT flags are not set, the TCE is dispositioned as an FP with a stellar eclipse major flag set. See \S\ref{s:secondTce}.

\textbf{INCONSISTENT\_TRANS}: The ratio of the maximum SES value to the MES value is above threshold and the TCE has a period greater than 90 days. This flag indicates that the TCE has only a few transits and the MES is dominated by a single large event. Thus, the TCE is dispositioned as an FP with the not transit-like major flag set. See \S\ref{s:sesmes}.

\textbf{INDI\_TRANS\_(CHASES|MARSHALL|\\SKYE|ZUMA|TRACKER|RUBBLE)}: One or more of the individual transit metrics (Chases, Marshall, Skye, Zuma, Tracker, or Rubble) removed a transit causing the TCE's recalculated MES to drop below threshold, or the number of transits to drop below 3. The TCE is dispositioned as an FP with the not transit-like major flag set. See \S\ref{s:indivtrans}.

\textbf{IS\_SEC\_TCE}: The TCE has the same period, but a different epoch, as a previous transit-like TCE on the same target. This indicates that the current TCE corresponds to the secondary eclipse of an eclipsing binary (or a planet if the PLANET\_OCCULT\_DV or PLANET\_OCCULT\_ALT flags are set). Thus, the current TCE is dispositioned as an FP with both the not transit-like and stellar eclipse major flags set. See \S\ref{s:issecond}.

\textbf{LPP\_(ALT|DV)}: The Locality Preserving Projections (LPP) value\citet{Thompson2015b}, as computed using the ALT or DV detrending, is above threshold. This indicates that the TCE is not transit-shaped, and thus is dispositioned as an FP with the not transit-like major flag set. See \S\ref{s:lpp}.

%\textbf{LPP\_DV}: The Locality Preserving Projections (LPP) value \citet{Thompson2015b}, as computed using the DV detrending, is above threshold. This indicates that the TCE is not transit-shaped, and thus is dispositioned as an FP with the not transit-like major flag set.

\textbf{MOD\_NONUNIQ\_(ALT|DV)}: The Model-shift 1 test, performed with the ALT or DV detrending, is below threshold. This test calculates the significance of the primary event, taking into account red noise, and compares it to the false alarm threshold. This flag indicates the primary event is not significant compared to the amount of systematic noise in the light curve, and thus the TCE is dispositioned as an FP with the not transit-like major flag set. See \S\ref{s:ms}.

%\textbf{MOD\_NONUNIQ\_DV}: The Model-shift 1 test, performed with the DV detrending, is below threshold. This test calculates the significance of the primary event, taking into account red noise, and compares it to the false alarm threshold. This flag indicates the primary event is not significant compared to the amount of systematic noise in the light curve, and thus the TCE is dispositioned as an FP with the not transit-like major flag set.

\textbf{MOD\_ODDEVEN\_(ALT|DV)}: The odd/even statistic from the Model-shift test is calculated with the ALT or DV detrending. This statistic compares the best-fit transit model to the odd and even transits separately and determines that the difference in the resulting significance values is above threshold. When set, the transit-like TCE is dispositioned as an FP with the stellar eclipse major flag set. See \S\ref{s:oddeven}.

%\textbf{MOD\_ODDEVEN\_DV}: The odd/even statistic from the Model-shift test is calculated with the DV detrending. This statistic compares the best-fit transit model to the odd and even transits separately and determines that the difference in the resulting significances is above threshold. When set, the transit-like TCE is dispositioned as an FP with the stellar eclipse major flag set.

\textbf{MOD\_POS\_(ALT|DV)}: The Model-shift 3 test, performed with the ALT or DV detrending, is below threshold. This test compares the significance of the primary and positive-going events in the phased light curve to help determine whether the primary event is unique. This flag indicates that the TCE is likely noise and thus is dispositioned as an FP with the not transit-like major flag set. See \S\ref{s:ms}.

%\textbf{MOD\_POS\_DV}: The Model-shift 3 test, performed with the DV detrending, is below threshold. This test compares the significance of the primary and positive-going events in the phased light curve to help determine whether the primary event is unique. This flag indicates that the TCE is likely noise and thus is dispositioned as an FP with the not transit-like major flag set.

\textbf{MOD\_SEC\_(ALT|DV)}: The Model-shift 4, 5, and 6 values, calculated using the ALT or DV detrending, are above threshold. This test calculates the significance of the secondary event divided by F$_{\mathrm{red}}$, the ratio of red noise to white noise in the light curve. The same calculation is done for the difference between the secondary and tertiary event significance values, and the difference between the secondary and positive event significance values. They indicate that there is a unique and significant secondary event in the light curve (i.e., a secondary eclipse). Thus, assuming the PLANET\_OCCUL\_(ALT|DV) flag is not set, the TCE is dispositioned as an FP with the stellar eclipse major flag set. See \S\ref{s:second}.

%\textbf{MOD\_SEC\_DV}: The Model-shift 4, 5, and 6 values, calculated using the DV detrending, are above threshold. This test calculates the significance of the secondary event divided by Fred, the ratio of red noise to white noise in the light curve. The same calculation is done for the difference between the secondary and tertiary event significances, and the difference between the secondary and positive event significances. They indicate that there is a unique and significant secondary event in the light curve (i.e., a secondary eclipse). Thus, assuming the PLANET\_OCCULT\_DV flag is not set, the TCE is dispositioned as an FP with the stellar eclipse major flag set.

\textbf{MOD\_TER\_(ALT|DV)}: The Model-shift 2 test, performed with the ALT or DV detrending, is below threshold. This test calculates the difference between the primary and tertiary event significance values. This flag indicates that the primary event is not unique in the phased light curve, and thus the TCE is likely noise and dispositioned as an FP with the not transit-like major flag set. See \S\ref{s:ms}.

%\textbf{MOD\_TER\_DV}: The Model-shift 2 test, performed with the DV detrending, is below threshold. This test calculates the difference between the primary and tertiary event significance values. This flag indicates that the primary event is not unique in the phased light curve, and thus the TCE is likely noise and dispositioned as an FP with the not transit-like major flag set.

\textbf{NO\_FITS}: Both the trapezoidal and the original DV transit fits failed to converge. This indicates the signal is not sufficiently transit-shaped in either detrending to be fit by a transit model. The TCE is dispositioned as an FP with the not transit-like major flag set. See \S\ref{s:nofits}.

\textbf{PERIOD\_ALIAS\_(ALT|DV)}: Using the phases of the primary, secondary, and tertiary events from the Model-shift test run on the ALT or DV detrended data, a possible period alias is seen at a ratio of $N$:1, where $N$ is an integer of 3 or greater. This indicates the TCE has likely been detected at a period that is $N$ times longer than the true orbital period. This flag is currently informational only and not used to declare any TCE an FP. See \S\ref{s:periodalias}.

%\textbf{PERIOD\_ALIAS\_DV}: Using the phases of the primary, secondary, and tertiary events from the Model-shift test run on the DV detrended data, a possible period alias is seen at n:1, where n is an integer of 3 or greater. This indicates the TCE has likely been detected at a period that is n times longer than the true orbital period. This flag is currently informational only and not used to declare any TCE an FP.

\textbf{PLANET\_IN\_STAR}: The original DV planet fits indicate that the fitted semi-major axis of the planet is smaller than the stellar radius. As it is possible that the stellar data is not accurate, this flag is currently informational only and not used to declare any TCE an FP. See \S\ref{s:planetinstar}.

\textbf{PLANET\_OCCULT\_(ALT|DV)}: A significant secondary eclipse was detected in the ALT or DV detrending, but it was determined to possibly be due to planetary reflection and/or thermal emission. While the stellar eclipse major flag remains set, the TCE is dispositioned as a PC. See \S\ref{s:sscand}.

%\textbf{PLANET\_OCCULT\_DV}: A significant secondary eclipse was detected in the DV detrending, but it was determined to possibly be due to planetary reflection and/or thermal emission. While the stellar eclipse major flag remains set, the TCE is dispositioned as a PC.

\textbf{PLANET\_PERIOD\_IS\_HALF\_(ALT|DV)}: A significant secondary eclipse was detected in the ALT or DV detrending, but it was determined to be the same depth as the primary within the uncertainties. Thus, the TCE is possibly a PC that was detected at twice the true orbital period. When this flag is set, it acts as an override to other flags such that the stellar eclipse major flag is not set, and thus the TCE is dispositioned as a PC if no other major flags are set. See \S\ref{s:sscand}.

%\textbf{PLANET\_PERIOD\_IS\_HALF\_DV}: A significant secondary eclipse was detected in the DV detrending, but it was determined to be the same depth as the primary within the uncertainties. Thus, the TCE is possibly a PC that was detected at twice the true orbital period. When this flag is set, it acts as an override to other flags such that the stellar eclipse major flag is not set, and thus the TCE is dispositioned as a PC if no other major flags are set.

\textbf{RESIDUAL\_TCE}: The TCE has the same period and epoch as a previous transit-like TCE. This indicates the current TCE is simply a residual artifact of the previous TCE that was not completely removed from the light curve. Thus, the current TCE is dispositioned as an FP with the not transit-like major flag set. See \S\ref{s:sameperiod}.

\textbf{SAME\_NTL\_PERIOD}: The current TCE has the same period as a previous TCE that was dispositioned as an FP with the not transit-like major flag set. This indicates that the current TCE is due to the same not transit-like signal. Thus, the current TCE is dispositioned as an FP with the not transit-like major flag set. See \S\ref{s:sameperiod}.

\textbf{SEASONAL\_DEPTH\_(ALT|DV)}: There appears to be a significant difference in the computed TCE depth from different seasons using the ALT or DV detrending. This indicates significant light contamination, usually due to a bright star at the edge of the aperture, which may or may not be the origin of the transit-like event. As it is impossible to determine whether or not the TCE is on-target from this flag alone, it is currently informational only and not used to declare any TCE an FP. See \S\ref{s:seasonaldiff}.

%\textbf{SEASONAL\_DEPTH\_DV}: There appears to be a significant difference in the computed TCE depth from different seasons using the DV detrending. This indicates significant light contamination, usually due to a bright star at the edge of the aperture, which may or may not be the origin of the transit-like event. As it is impossible to determine whether or not the TCE is on-target from this flag alone, it is currently informational only and not used to declare any TCE an FP.

\textbf{SWEET\_EB}: The sine wave event evaluation test (SWEET) is above threshold, the detected signal has an amplitude less than the TCE's depth, and the TCE period is less than 5 days. This flag indicates that there is a significant sinusoidal variability in the PDC data at the same period as the TCE due to out-of-eclipse EB variability. The transit-like TCE is dispositioned as an FP with the stellar eclipse major flag set. See \S\ref{s:sweeteb}.

\textbf{SWEET\_NTL}: The sine wave event evaluation test (SWEET) is above threshold, the detected signal has an amplitude greater than the TCE's depth, and the TCE period is less than 5 days. This flag indicates that there is a significant sinusoidal variability in the PDC data at the same period as the TCE, and the detected event is due to stellar variability and not a transit. The TCE is dispositioned as an FP with the not transit-like major flag set. See \S\ref{s:sweetntl}.

\textbf{TRANS\_GAPPED}: The fraction of gapped transit events is above threshold. This flag indicates that a large number of observable transits had insufficient in-cadence data. The TCE is dispositioned as an FP with the not transit-like major flag set. See \S\ref{s:rocky}.
