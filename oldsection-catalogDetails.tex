%Catalog Details and Analysis

%\subsection{The physical parameters of the planets}

\subsection{Potentially Rocky Planets in the Habitable Zone}
This is a closer look at some of our most interesting objects.
\subsection{Known Issues}
\subsubsection{Confirmed False Positives}



\subsection{New Multi-planet systems}
%%From Jack Lissauer
Only XX, or YY per cent of the new planet candidates are associated with targets that have multiple planet candidates. 

One of the new planet candidates, K00082.06, is part of a six planet candidate system.  The other five candidates have been verified as planets, and this system is known as Kepler-102 \citep{Marcy2014,Rowe2014}.  The new candidate is a bit exterior to the 4:3 resonance with the largest verified planet in the system, and there is an excess of planets found just wide of such first-order resonances (Lissauer et al. 2011, ApJL), suggesting that this candidate is likely to be a planet.  If verified, this would be only the 3rd system with six or more planets found by Kepler. 
The other new candidate within a high multiplicity system is KOI-2926.05.  The other four candidates in this system have been validated as Kepler-1388 by \citet{Morton2016}.  This new candidate also orbits just exterior to a first-order mean motion resonance with one of the 4 previously known planets, again adding to the likelihood that it is a true planet. 

%This percentage is smaller than the percentage of planet candidates in the previous catalog that reside in multiple planet systems.  This decrease is probably related to the increased fraction of long-period candidates among the newly-identified population, as such planets are less likely to be found in multiple planet systems \citep{Lissauer2014}.