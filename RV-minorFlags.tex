\section{Robovetter Mnemonic Flags}
\label{minorflagsec}
\red{This needs to be updated with DR25 mnemonic flags.}
In Table~\ref{robodispstab} we list mnemonic flags that describe the results of individual robovetter tests in the comments column. Here we describe the meaning of each flag.

\begin{itemize}
\item[] \textbf{ALT\_ROBO\_ODD\_EVEN\_TEST\_FAIL}: The TCE failed the robovetter's odd-even depth test on the alternate detrending, and thus is marked as a FP due to a significant secondary.
\item[] \textbf{ALT\_SEC\_COULD\_BE\_DUE\_TO\_PLANET}: A significant secondary eclipse was detected in the alternate detrending, but it was determined to possibly be due to planetary reflection and/or thermal emission. While the significant secondary major flag remains set, the TCE is dispositioned as a PC.
\item[] \textbf{ALT\_SEC\_SAME\_DEPTH\_AS\_PRI\_COULD\_BE\_TWICE\_TRUE\_PERIOD}: A significant secondary eclipse was detected in the alternate detrending, but it was determined to be the same depth as the primary within the uncertainties. Thus, the TCE is possibly a PC that was detected at twice the true orbital period. When this flag is set, it acts as an override to other flags such that the significant secondary major flag is not set, and thus the TCE is dispositioned as a PC if no other major flags are set.
\item[] \textbf{ALT\_SIG\_PRI\_MINUS\_SIG\_POS\_TOO\_LOW}: The difference of the primary and positive event significances, computed by the model-shift test using the alternate detrending, is below the threshold $\sigma'_{\rm FA}$. This indicates the primary event is not unique in the phased light curve, and thus the TCE is dispositioned as a FP with the not transit-like major flag set.
\item[] \textbf{ALT\_SIG\_PRI\_MINUS\_SIG\_TER\_TOO\_LOW}: The difference of the primary and tertiary event significances, computed by the model-shift test using the alternate detrending, is below the threshold $\sigma'_{\rm FA}$. This indicates the primary event is not unique in the phased light curve, and thus the TCE is dispositioned as a FP with the not transit-like major flag set.
\item[] \textbf{ALT\_SIG\_PRI\_OVER\_FRED\_TOO\_LOW}: The significance of the primary event divided by the ratio of red noise to white noise in the light curve, computed by the model-shift test using the alternate detrending, is below the threshold $\sigma_{\rm FA}$. This indicates the primary event is not significant compared to the amount of systematic noise in the light curve, and thus the TCE is dispositioned as a FP with the not transit-like major flag set.
\item[] \textbf{CENTROID\_SIGNIF\_UNCERTAIN}: The significance of the centroid offset cannot be measured to high enough precision, and thus the centroid module can not confidently disposition the TCE as a FP. This is typically due to having only a very small number (3 or 4) of offset measurements, all with low SNR.
\item[] \textbf{CLEAR\_APO}: The TCE was marked as a FP due to a centroid offset because the transit occurs on a star that is spatially resolved from the target.
\item[] \textbf{CROWDED\_DIFF}: More than one potential stellar image was found in the difference image. The EYEBALL flag is always set when the CROWDED\_DIFF flag is set.
\item[] \textbf{DV\_ROBO\_ODD\_EVEN\_TEST\_FAIL}: The TCE failed the robovetter's odd-even depth test on the DV detrending, and thus is marked as a FP due to a significant secondary.
\item[] \textbf{DV\_SEC\_COULD\_BE\_DUE\_TO\_PLANET}: A significant secondary eclipse was detected in the DV detrending, but it was determined to possibly be due to planetary reflection and/or thermal emission. While the significant secondary major flag remains set, the TCE is dispositioned as a PC.  
\item[] \textbf{DV\_SEC\_SAME\_DEPTH\_AS\_PRI\_COULD\_BE\_TWICE\_TRUE\_PERIOD}: A significant secondary eclipse was detected in the DV detrending, but it was determined to be the same depth as the primary within the uncertainties. Thus, the TCE is possibly a PC that was detected at twice the true orbital period. When this flag is set, it acts as an override to other flags such that the significant secondary major flag is not set, and thus the TCE is dispositioned as a PC if no other major flags are set.
\item[] \textbf{DV\_SIG\_PRI\_MINUS\_SIG\_POS\_TOO\_LOW}: The difference of the primary and positive event significances, computed by the model-shift test using the DV detrending, is below the threshold $\sigma'_{\rm FA}$. This indicates the primary event is not unique in the phased light curve, and thus the TCE is dispositioned as a FP with the not transit-like major flag set.
\item[] \textbf{DV\_SIG\_PRI\_MINUS\_SIG\_TER\_TOO\_LOW}: The difference of the primary and tertiary event significances, computed by the model-shift test using the DV detrending, is below the threshold $\sigma'_{\rm FA}$. This indicates the primary event is not unique in the phased light curve, and thus the TCE is dispositioned as a FP with the not transit-like major flag set.  
\item[] \textbf{DV\_SIG\_PRI\_OVER\_FRED\_TOO\_LOW}: The significance of the primary event divided by the ratio of red noise to white noise in the light curve, computed by the model-shift test using the DV detrending, is below the threshold $\sigma_{\rm FA}$. This indicates the primary event is not significant compared to the amount of systematic noise in the light curve, and thus the TCE is dispositioned as a FP with the not transit-like major flag set.  
\item[] \textbf{EYEBALL}: The metrics used by the centroid module are very close to the decision boundaries, and thus the centroid disposition of this TCE is uncertain and warrants further scrutiny. No TCEs are marked as a FP due to a centroid offset if this flag is set.
\item[] \textbf{FIT\_FAILED}: The transit was not fit by a model in DV and thus no difference images were created for use by the centroid module. Thus, the TCE is not failed due to a centroid offset by default. This flag is typically set for very deep transits due to eclipsing binaries.
\item[] \textbf{INVERT\_DIFF}: One or more difference images were inverted, meaning the difference image claims the star got brighter during transit. This is usually due to variability of the target star and suggests the difference image should not be trusted. When this flag is set, the TCE is marked as a candidate that requires further scrutiny, i.e., the EYEBALL flag is set and the TCE is not marked as a FP due to a centroid offset.
\item[] \textbf{KIC\_OFFSET}: The centroid module measured the offset distance relative to the star's recorded position in the Kepler Input Catalog (KIC), not the out of transit centroid. The KIC position is less accurate in sparse fields, but more accurate in crowded fields. If this is the only flag set, there is no reason to believe a statistically significant centroid shift is present \citep{Mullally2015c}.
\item[] \textbf{LPP\_ALT\_TOO\_HIGH}: The LPP value \citep{Thompson2015b}, as computed using the alternate detrending, is above the robovetter threshold. This indicates the TCE is not transit-shaped, and thus is dispositioned as a FP with the not transit-like major flag set.
\item[] \textbf{LPP\_DV\_TOO\_HIGH}: The LPP value, as computed using the DV detrending, is above the robovetter threshold. This indicates the TCE is not transit-shaped, and thus is dispositioned as a FP with the not transit-like major flag set.  
\item[] \textbf{MARSHALL\_FAIL}: The TCE failed the Marshall metric \citep{Mullally2015b}, which indicates that the TCE's individual transits are not transit-shaped and more likely due to instrumental artifacts. Thus, the TCE is dispositioned as a FP with the not transit-like major flag set.
\item[] \textbf{OTHER\_TCE\_AT\_SAME\_PERIOD\_DIFF\_EPOCH}: Another TCE on the same target with a higher planet number was found to have the same period as the current TCE, but a significantly different epoch. This indicates the current TCE is an eclipsing binary with the other TCE representing the secondary eclipse. If the ALT\_SEC\_COULD\_BE\_DUE\_TO\_PLANET and DV\_SEC\_COULD\_BE\_DUE\_TO\_PLANET flags are not set, the TCE is dispositioned as a FP with the significant secondary major flag set.
\item[] \textbf{PARENT\_IS\_X}: The TCE has been identified as a FP due to an ephemeris match. This flag indicates the most likely parent, or true physical source of the signal, where X will be substituted for the parent's name. Note that X is not guaranteed to be the true parent, but simply is the most likely source given the information available.
\item[] \textbf{PERIOD\_ALIAS\_IN\_ALT\_DATA\_SEEN\_AT\_X:1}: Using the results of the model-shift test (specifically the phases of the primary, secondary, and tertiary events) a possible period alias is seen at X:1, where X is an integer. This indicates the TCE has likely been detected at a period that is X times longer than the true orbital period. This flag is currently informational only and not used to declare any TCE a FP.
\item[] \textbf{RESID\_OF\_PREV\_TCE}: The TCE has the same period and epoch as a previous transit-like TCE. This indicates the current TCE is simply a residual artifact of the previous TCE after it was removed from the light curve. Thus, the current TCE is dispositioned as a FP with the not transit-like major flag set.
\item[] \textbf{SAME\_P\_AS\_PREV\_NTL\_TCE}: The current TCE has the same period as a previous TCE that was dispositioned as FP with the not transit-like major flag set. This indicates that the current TCE is due to the same not transit-like signal. Thus, the current TCE is dispositioned as a FP with the not transit-like major flag set.
\item[] \textbf{SATURATED}: The star is saturated. The assumptions employed by the centroid robovetter module break down for saturated stars, so the TCE is marked as a candidate requiring further scrutiny, i.e., the EYEBALL flag is set and the TCE is not marked as a FP due to a centroid offset.
\item[] \textbf{SEASONAL\_DEPTH\_DIFFS\_IN\_ALT}: There appears to be a significant difference in the computed TCE depth when using the alternate detrending light curves from different seasons. This indicates significant light contamination is present, usually due to a bright star at the edge of the image, which may or may not be the source of the signal. As it is impossible to determine whether or not the TCE is on-target from this flag alone, it is currently informational only and not used to declare any TCE a FP.
\item[] \textbf{SEASONAL\_DEPTH\_DIFFS\_IN\_DV}: There appears to be a significant difference in the computed TCE depth when using the DV detrending light curves from different seasons. This indicates significant light contamination is present, usually due to a bright star at the edge of the image, which may or may not be the source of the signal. As it is impossible to determine whether or not the TCE is on-target from this flag alone, it is currently informational only and not used to declare any TCE a FP.  
\item[] \textbf{SIG\_SEC\_IN\_ALT\_MODEL\_SHIFT}: The significance of the secondary event divided by the ratio of red noise to white noise in the light curve, computed by the model-shift test using the alternate detrending, is above the threshold $\sigma_{\rm FA}$. Also, the difference between the secondary and tertiary event significances, and the difference between the secondary and positive event significances, both computed by the model-shift test using the alternate detrending, is above the threshold $\sigma'_{\rm FA}$. This indicates that there is a unique and significant secondary event in the light curve, i.e., a secondary eclipse. Thus, assuming the ALT\_SEC\_COULD\_BE\_DUE\_TO\_PLANET flag is not set, the TCE is dispositioned as a FP with the significant secondary flag set.
\item[] \textbf{SIG\_SEC\_IN\_DV\_MODEL\_SHIFT}: The significance of the secondary event divided by the ratio of red noise to white noise in the light curve, computed by the model-shift test using the DV detrending, is above the threshold $\sigma_{\rm FA}$. Also, the difference between the secondary and tertiary event significances, and the difference between the secondary and positive event significances, both computed by the model-shift test using the DV detrending, is above the threshold $\sigma'_{\rm FA}$. This indicates that there is a unique and significant secondary event in the light curve, i.e., a secondary eclipse. Thus, assuming the DV\_SEC\_COULD\_BE\_DUE\_TO\_PLANET flag is not set, the TCE is dispositioned as a FP with the significant secondary flag set.
\item[] \textbf{SIGNIF\_OFFSET}: There is a statistically significant shift in the centroid during transit. This indicates the variability is not due to the target star. Thus, the TCE is dispositioned as a FP with the centroid offset major flag set.
\item[] \textbf{THIS\_TCE\_IS\_A\_SEC}: The TCE is determined to have the same period, but different epoch, as a previous transit-like TCE. This indicates that the current TCE corresponds to the secondary eclipse of an eclipsing binary (or planet if the ALT\_SEC\_COULD\_BE\_DUE\_TO\_PLANET or DV\_SEC\_COULD\_BE\_DUE\_TO\_PLANET flags are set.) Thus, the current TCE is dispositioned as a FP with both the not transit-like and significant secondary major flags set.
\item[] \textbf{TOO\_FEW\_CENTROIDS}: The PRF centroid fit used by the centroid module does not always converge, even in high SNR difference images. This flag is set if centroid offsets are recorded for fewer than 3 high SNR difference images.
\item[] \textbf{TOO\_FEW\_QUARTERS}: Fewer than 3 difference images of sufficiently high SNR are available, and thus very few tests in the centroid module are applicable to the TCE. If this flag is set in conjunction with the CLEAR\_APO flag, the source of the transit may be on a star clearly resolved from the target.
\item[] \textbf{TRANSITS\_NOT\_CONSISTENT}: The TCE had a max\_ses\_in\_mes / mes ratio of greater than 0.9, and a period greater than 90 days. This indicates that the TCE is dominated by a single large event, and thus is due to a systematic feature such as a sudden pixel sensitivity dropout. Thus, the TCE is dispositioned as a FP with the not transit-like major flag set.
\end{itemize}
