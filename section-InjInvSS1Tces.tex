\subsection{Simulated TCEs}
\label{s:simulated}
Because we needed a way to identify when the Robovetter was not performing well, and because we wanted to be able to measure the performance of the Robovetter, we created simulated transits and simulated false alarms. The simulated transits are created by the injecting transit signals onto the pixels of our original data.  The simulated false alarms caused by instrumental and stellar noise were created in two ways, 1) by inverting our light curve and 2) by scrambling the time order of the time series.  


\subsubsection{True Transits -- Injection}
\label{injectsec}
[NOT DONE: JESSIE WILL WRITE SOMETHING HERE\newline]

The pixel-level transit injection is similar to that used by DR24 \citep{Christiansen2016} and is described in detail for DR25 in \citep{Christiansen2017}. The pixel-level injection study provides several types of injections: those on the target star, those offset from the location of the target star, two injections on the same star to emulate an eclipsing binary, and a special period range for the M-dwarf stars.  For our purposes in this paper, we will focus on the on-target injections of all type.  These injected signals were placed on the calibrated pixel data and sent through the same \Kepler{} pipeline as the \opstce s.


\subsubsection{False Alarms -- Inverted and Scrambled} 

To create realistic false alarms which have noise properties similar to our \opstce{s}, we inverted the normalized light curve (i.e. multiplied it by negative one) before searching it for transit signals.  Because the pipeline is only looking for transit-like dips in the light curve, the exoplanet transits are no longer found. However, periodic instrumental and stellar noise are found. This approach assumes that the sources of false alarms are symmetric in relative flux. The period distribution of these \invtce s is shown in Figure~\ref{f:tces} as the blue histogram. 
The distribution across period of these events basically emulates those seen in the \opstce s; however there are only about 60 per cent as many.  The one-year spike is clearly seen, but again is not as tall as we might expect given the size of the one-year spike in the \opstce\ period distribution. 

Another method to create false alarms is to scramble the order of the data.  The trick is to scramble the data enough to lose the coherence for the binary stars and the exoplanets in the data, but keep some of the coherency of the red noise that plagues the \Kepler\ data set.  We accomplished this by scrambling the data by year. In other words we moved year 4 to the front, followed by year 3, then 2, then 1. Q17 was kept at the end of the time series. Within each year, the order of the data did not change. The final order of the quarters in the scrambled data set was 13,14,15,16,9,10,11,12,5,6,7,8,1,2,3,4,17.  Notice that in this configuration each quarter remains in the correct \Kepler\ season, and so some of the yearly, rolling-band artifacts create false alarms in this way \citep{KIH}.



%In Figure~\ref{f:false} we show an example of the folded light curves of an \opstce, \invtce, and \scrtce\ where the number of identified transit events is three.  Notice that for low signal-to-noise \Kepler\ data is capable of having three events due to noise line-up and produce something that looks very similar to an exoplanet transit.  

\subsubsection{Cleaning Inversion and Scrambling}
\label{s:clean}
As described in \S\ref{s:relcalc}, we want to use the \invtce\ and \scrtce\ sets to measure the reliability of the DR25 catalog against instrumental and stellar noise.  In order to do that well, we need to remove signals found in these sets that are not typical of those in our \opstce\ set.  For inversion there are astrophysical events that look similar to an inverted eclipse, for example the self-lensing binary star, K03278.01 \citep{Kruse2014}, and Heartbeat Binaries \citep{Thompson2012}.  With the assistance of published systems and early runs of the Robovetter, we identified any \invtce\ that could be identified as possibly one of these types of astrophysical events; 54 systems were identified in total.  Also, the shoulders of inverted eclipsing binary stars and high signal-to-noise KOIs were found by the pipeline but are not the type of false alarm we were trying to reproduce.  We remove any \invtce\ that was found on a star that had 1) one of the identified astrophysical events, 2) detached eclipsing binaries listed in \citet{Kirk2016} with morphology values larger than 0.6, and 3) a known KOI.  This results in \ninvtces\ \invtce s and their distribution is plotted in Figure~\ref{f:tces}.

For season scrambling, we do not have to worry about the astrophysical events that emulate an inverted transit, but we do have to worry about triggering on true transits. For this reason we removed from the \scrtce\ population any one that landed on a star with a known eclipsing binary \citep{Kirk2016}, or on a previously identified KOI.  The result is \nscrtces\ \scrtce s; their distribution is plotted in Figure~\ref{f:tces}. 

After cleaning the \invtce s and \scrtce s of the eclipsing binary and KOI systems, the number of \scrtce s at periods longer than 200\,d closely matches the size and shape of the same period range in the \opstce s, except for the one-year spike.  The one-year spike is well represented by the \invtce s.  Combining the two sets appears to give us a good handle on the type and relative frequency of false alarms present in our \opstce\ population. Table\,\ref{tab:invscr} lists those \invtce s and \scrtce s that we used when calculating the effectiveness and thus reliability of the catalog.

%\subsection{How good are the simulated data sets}
%We should show some evidence that the fraction of false alarms in the simulated data sets matches those in the actual OBS data.  Can we show that the fraction of fails by LPP, individual transits and sig-pri/Fred are approximately the same between the two sets.  Or how different are they.  Do this for low mes,

[MISSING: Table Listing those KICs we cleaned away]

%\subsubsection{Other Simulated sets}
%We also simulated other types of false positives through transit injection \citep{Christiansen2017}. A portion of the injections were injected off the target by less than a pixel.  This is a way of testing our ability to find background eclipsing binaries by looking for small shifts in the location of the transit compared to the location of the target star.  Also, we injected two transits on some stars to emulate an eclipsing binary and test our tools to identify the presence of a significant secondary eclipse.  However because the majority of our false positives are caused by instrumental noise, and because the size of the astrophysical false positive reliability has been well characterized for \Kepler\ \citep[e.g.][]{Morton2016}, we limit our reliability discussion to those caused by instrumental and stellar noise. 


%All of the data sets described above are available in their entirety at the NASA Exoplanet Archive.  All of the metrics described in \S\ref{s:robovetter} to evaluate the disposition of these TCEs are also available in the same table. 