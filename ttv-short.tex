\begin{deluxetable}{crrr}
\tabletypesize{\scriptsize}
\tablecaption{TTV Measurements of KOIs}
\tablewidth{0pt}
\tablehead{
\colhead{$n$}  & \colhead{$t_n$} & \colhead{$TTV_n$}  & \colhead{$TTV_{n\sigma}$} \\
\colhead{}     & \colhead{BJD-2454900.0}      & \colhead{days}     & \colhead{days}
}
\startdata
KOI-6.01 & & & \\
1 & 54.6961006 & 0.0774247 &  0.0147653 \\ 
2 & 56.0302021 & -0.0029102 &  0.0187065 \\ 
3 & 57.3643036 & -0.0734907 &  0.0190672 \\ 
4 & 58.6984051 & 0.0119630 &  0.0176231 \\ 
\nodata & \nodata & \nodata & \nodata\\
KOI-8.01 & & & \\
1 & 54.7046603 & -0.0001052 &  0.0101507 \\ 
2 & 55.8648130 & -0.0103412 &  0.0084821 \\ 
3 & 57.0249656 & 0.0047752 &  0.0071993 \\ 
\nodata & \nodata & \nodata & \nodata\\
KOI-8151.01 & & & \\
1 & 324.6953389 & 0.1093384 &  0.0025765 \\ 
2 & 756.2139285 & -0.3478332 &  0.0015206 \\ 
3 & 1187.7325181 & 0.0110542 &  0.0016480 \\
\nodata & \nodata & \nodata & \nodata\\
\enddata
\tablecomments{Column 1, $n$, is the transit number. Column 2, $t_n$, is the transit time in Barycentric Julian Date minus the offset 2454900.0. Column 3, $TTV_n$, is the observed - calculated (O-C) transit time. Column 4, $TTV_{n\sigma}$,  is the $1\sigma$ uncertainty in the O-C transit time.
Table \ref{t:ttv} is published in its entirety in the electronic edition of the {\it Astrophysical Journal Supplement}.  A portion is shown here for guidance regarding its form and content.}
\label{t:ttv}
\end{deluxetable}
