%Section Balancing Completeness and Reliability
\label{s:optimize}
As described in the previous section, the Robovetter makes decisions on which TCEs are FPs and PCs based on a collection of metrics and thresholds.  For each metric we apply a threshold and if the TCE's metric's value lies above (or below, depending on the metric) the TCE is called a false positive.  No metric is perfect.  The metrics we use in the Robovetter are those that were identified to remove a significant number of FPs while only throwing away a small number of PCs.  As a result the set of potential thresholds to use is infinite and there is no perfect way to tune the Robovetter.

Our tuning of the Robovetter relied heavily on the simulated data sets described \S\ref{s:simulated}. The \injtce\ set gives us a population of true transit signals and the completeness of the Robovetter is given by the fraction of these that are created into PCs.  The \scrtce\ and \invtce\ set gives us a way to measure the effectiveness ($E$) of the Robovetter at removing those types of false positives that are emulated by these sets. As discussed in \S\ref{s:relcalc} with both $E$ and the number of \opstce s that are turned into PCs with those same metrics, we can measure the reliability of those observed PCs.  

We tuned the Robovetter primarily by measuring the reliability and completeness measured from these simulated data sets. 

\subsection{Setting Metric Thresholds Through Optimization}
\label{s:full_optimize}
For the first step in Robovetter tuning, we perform an optimization that finds the metric thresholds that maximize the number TCEs from the \injtce\ set classified as PCs, called {\it true positives}, and minimizes the number of TCEs from the \scrtce\ and \invtce\ sets identified as PCs, called {\it false positives}.  We define the {\it true positive fraction} as the ratio of the number of \injtce\ TCEs classified as PC to the total number of \injtce\ TCEs, and the {\it false positive fraction} as the number of TCEs from the \scrtce\ and \invtce\ sets identified as PCs to the total number of \scrtce\ and \invtce\ TCEs.  Optimization varies the thresholds of the subset of Robovetter metrics described below, looking for those metric thresholds that maximizes the true positive fraction and minimizes the false positive fraction.  [CAN WE CHANGE THIS TO COMPLETENESS AND EFFECTIVENESS INSTEAD OF CREATING NEW TERMS?]

This optimization is performed jointly across a subset of the metrics described in \S\ref{s:robovetter}.  Metrics chosen for this joint optimization are either believed {\it a priori} to distinguish true positives from false positives or have distributions in the \injtce\ set are statistically well-separated from those in the \scrtce\ and \invtce\ sets.  By ``well-separated" we mean that the distributions' medians are separated by more more than 0.5 times the largest maximum absolute deviation of the \injtce, \scrtce\ or \invtce\ sets.  The set of metrics chosen for the joint optimization, called ``optimized metrics" are: LPP for the DV and alternate detrending (\S\ref{s:lpp}), the model shift uniqueness test metrics $MS_{1}$, $MS_{2}$, and $MS_{3}$ for the DV and alternate detrending (\S\ref{s:ms}), ${{\rm SES}_{\rm Max} / {\rm MES}}$ (\S\ref{s:sesmes}), and sesArt (what's that?). 

Metrics not used in the joint optimization are incorporated by classifying TCEs as PCs or FPs using {\it a priori} thresholds for these non-optimized metrics prior to optimization of the optimized metrics.  After optimization, a TCE is classified as a PC only if it passes both the non-optimized metrics and the optimized metrics.  Prior to optimization the thresholds for these non-optimized metrics passed about 80\% of the  \injtce\ set, so the final optimized set can have at most 80\% completeness.  

Optimization is performed by varying the optimized thresholds, determining which TCEs are classified as PCs by both the optimized and non-optimized metrics using the new optimized thresholds, computing the TPF and FPF.  Our optimization seeks thresholds that minimize the objective function $\sqrt{{\rm FPF}^2 + ({\rm TPF} - {\rm TPF}_0)^2}$, where ${\rm TPF}_0$ is the target true positive fraction, so the optimization tries to get as close as possible to ${\rm FPF} = 0$ and ${\rm TPF} = {\rm TPF}_0$.  We varied ${\rm TPF}_0$ in an effort to minimize the false positive fraction. The thresholds are varied from random starting seed values, using the Nelder-Mead simplex algorithm via the MATLAB {\it fminsearch} function.  This MATLAB function varies the thresholds until the objective function is minimized.  There are many local minima, so the optimal thresholds depend sensitively on the random starting threshold values.  The optimal thresholds we use provide the smallest  of 2000 optimizations with different random seed values.

Our final optimal threshold used ${\rm TPF}_0 = 0.8$, which resulted in thresholds that gave ${\rm FPF} = 0.0044$ and ${\rm TPF} = 0.799$.  We experimented with smaller values of ${\rm TPF}_0$, but these did not significantly lower FPF.   We also performed an optimization that maximizes reliability defined in \S\ref{s:relcalc} rather than minimizes FPF, with similar results.  

We explored using only one of \scrtce\ or \invtce\ to determine the FPF, and found that using both provided the best starting point for the final Robovetter thresholds described below.  We also explored the dependence of the optimal thresholds on the range of TCE MES and period.  We found that the thresholds have a moderate dependence, while the FPF and TPF have significant dependence on MES and period range.  For example, optimizing to the low-MES, long-period regime of long-period small-planet TCEs,  we find ${\rm FPF} = 0.028$ and ${\rm TPF} = 0.73$.  Exploration of this dependence of Robovetter threshold on MES and period range is a topic of future study.


\subsection{Picking a Robovetter}
Here we describe how we went about deciding on the thresholds we use in this DR25 catalog
[CAN JEFF WRITE A QUICK PARAGRAPH HERE MOTIVATING WHY WE DIDN'T USE THE ABOVE]\\
[DO WE WANT A TABLE OF THE EXACT THRESHOLDS STEVE FOUND?]\\
