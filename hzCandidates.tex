We take a closer look at some of the HZ candidates. All of those in this list have a 1 sigma error bar that places them below 1.8 Rearth and less than 2.0 Searth according to the MCMC fits, has an relative error in Rp less than 33 per cent and has a score cut larger than 0.6.

\paragraph K08279.01 on \opstce\,009011955-01 
This TCE is dispositioned as PC with a score of 0.7 but when inspected manually, it is clearly caused by Rolling band. The fit gives a radius of 5.22, but it has large uncertainties which give it some possibility of being terrestrial. However this is clearly an FP. 

\paragraph K08174.01 on \opstce\,008873873-01
This candidate is new to DR25 and has five transits, three of which show a significant reduction in flux. It is measured to have a radii smaller than the Earth's and receives slightly less insolation flux than the Earth. It is around a 5300\,K star. It has low MES. It likely has an anomolously small radius.

\paragraph K08007.01 on \opstce\,010403228-03
This candidate is new to DR25 and has a score of 0.802. There are only five quarters of observations for this star (Q8 and Q14--Q17), so this 67\,d period TCE only has 4 transits. None of these transits show a clear signal, so this is a marginal PC.

\paragraph K08000.01 on \opstce\,010331279-01
This candidate is new to DR25 and has five transits that do not lie in data gaps (score=0.974). It has a period of 225\,d and a MES of 8.7. It circles a star that is 5663. It's planet radii ranges from 1.52 to 2.07 Rearth. 

\paragraph K07923.01 on \opstce\,009084569-02
This candidate is new to DR25 and has 5 transits. Only one or two of these transits truly look transit-shaped. Marshall only preferred a transit shape for one, but not with enough significance to actually throw away most of those transits. Likely this is an FP.

\paragraph K07894.01 on \opstce\,008555967-01
This candidate is new to DR25 and has four transits. The Q5 transit looks more like a step-up than a transit, though Marshall preferred the sigmoid-box model for that transit. It has a MES of 8.5 and a period of 347.976\,days. Our MCMC model gives this a radius of 1.79 (-0.15 +0.49) and an insolation flux of 0.97 (-0.27, +0.87), i.e. likely a super-Earth with an insolation flux similar to the Earth's.

\paragraph K07749.01 on \opstce\,005888187-01
This candidate is new to DR25 and has 10 transits. It has a MES of 8.2 and a period of 133\,d. While it is a low signal to noise candidate, there is not reason to question it. It would have a radius of 1.89 and insolation flux of 2.38 around a 5202\,K star (Between and K and G dwarf)

\paragraph K07711.01 on \opstce\,004940203-01
This candidate is new to DR25 and has four transits. Each transit has no issues and three were passed cleanly by Marshall.  This planet candidate is in a 302\,d orbit with a MES of 8.4 and score of 0.99.  It  lies around a 5734\,K star and has an insolation flux of 0.87 (-.22,+0.66). The radii is 1.47 (-.12, +0.34).

\paragraph K07706.01 on \opstce\,004940203-01
This candidate is new to DR25 and has 23 good transits.  The star shows large flux variations likely due to spots, but the transit duration is significantly shorter than these spot variations and so likely the detrending is creating the signal. It has a period of 42.0495\,d and MES of 7.4 with a score of 0.84. It circles a 4281\,K star. It's radius s 1.63 (-.16, +0.08) with an insolation flux of 2.0 (-0.68 +0.55).


This is the list of new Candidates with score values less than 0.6.  These have a lower reliability (~40 per cent)

\paragraph K08290.01 on \opstce\,010459805-01
This FP has a score of 0.113. Two of its four transits are possible SPSDs (Q0 and Q7) and the remaining signal is not enough to convince one there is a transit here.

\paragraph K08278.01 on \opstce\,008873674-01
This is dispositioned as an FP with a score of 0.448. This fails because Marshall removes two of the transits (Q6 and Q15) because they are more consistent with no transit. This causes the calculated MES to drop a bit below 7.1. Given that it has 7 observed transits, by manual inspection it would be a marginal candidate. If so it has a radius of 1.82 and an insolation flux of 1.95 which places it as a potential super Earth in the habitable zone of this G type star.

\paragraph K08272.01 on \opstce\,008022520-01
This candidate has a score of 0.117 meaning that several metrics were on the verge of failing it. It has three transits of which one is truly transit shaped in the DV detrending, the alternate detrending looks pretty good.

\paragraph K08259.01 on \opstce\,010459805-01
This PC with a score of 0.33 has four transits. Q8 is has most of the signal and is associated with noise in the sky causing the transit to be quite noisy. The other three events are not convincing. This is likely an FP.

\paragraph K08253.01 on \opstce\,005098334-01
This PC has a score of 0.07. It has three transits, but none really look like transits. This is likely an FP.

\paragraph K08246.01 on \opstce\,004446859-01
This PC with a score of 0.223 has four transits on a pretty clean light curve. Other than the low score there is no reason to fail this candidate. The candidate orbits a 6067\,K G-type star at a 425\,d period orbit. It is 1.61 (-0.37, +0.45) radius and the insolation flux is 1.48.

\paragraph K08242.01 on \opstce\,003629119-01
This PC with a score of 0.184 is very low signal-to-noise but otherwise has no reason to fail it.  It has a radius of 1.82 and an insolation flux of 0.78. This is an FP.

\paragraph K08227.01 on \opstce\,011558217-01
This FP has a score of 0.221. This correctly fails because the events are different in shape and depth.

\paragraph K08085.01 on \opstce\,002284957-01
This PC has a score 0.421. From manual inspection it is clear that two of the transit events are actually SPSDs. The other two transit events are identified correctly by Marshall to be better fit by an offset only. This is likely an FP.

\paragraph K08077.01 on \opstce\,012266099-01
This PC has a score of 0.341. This is a marginal, low snr PC. At 4840\,K and 1.9 radii and 2.7 insolation flux, but the large error bars cause it to have some possibility of being in the HZ.

\paragraph K08048.01 on \opstce\,012266099-01
This PC has a score of 0.332. It is a marginal, low snr PC, but otherwise fine. It has a radii of 1.52 and insolation flux of 0.88 around a cool F-dwarf star (6156\,K)

\paragraph K07992.01 on \opstce\,010214341-01
This PC has a score of 0.124 with four transits. The transit in Q13 is associated with a sky flare and Q7, which contains most of the remaining flux, is not transit shaped. This is likely an FP. 

\paragraph K07953.01 on \opstce\,009650579-01
This PC has a score of 0.339 and three transits. None of the transits are very convincing, likely this is an FP.

\paragraph K07932.01 on \opstce\,009278575-01
This PC has a score of 0.229 and three transits. None of the transits are very convincing, like this is an FP. The strongest transit in Q12 is associated with noisy data points and a discontinuity in the sky time series.

\paragraph K07938.01 on \opstce\,009469494-01
This PC has a score of 0.510. It has an extremely large error bar in insolation flux that yields some possibility that it could lie in the habitable zone. A manual inspection shows that this is a highly variable star and the transits artifacts of less than optimal detrending. Likely this is an FP.

\paragraph K07930.01 on \opstce\,009210820-01
This PC has a score of 0.469 and is a low signal-to-noise transit with only 3 transits. Manual inspection yields no reason to doubt this candidate. Its radius is 50 percent larger than the earth's and circles a 5031\,K K-dwarf at a period similar to the Earth's causing it have an insolation flux similar to Mars.

\paragrph K07882.01 on \opstce\,008364232-01
This PC has a score of 0.552 and circles a star that is 4348\,K, a mid K-dwarf with a period of 65\,d. It has a MES of 7.2, making it very low signal to noise and thus marginally passes the Robovetter.  Manual inspection reveals that there is no reason to doubt this signal.

\paragraph K07798.01 on \opstce\,006952971-01
This PC has a score of 0.474. It is dominated by what looks like noise or a discontinuity in Q16 and the TCE is likely an FP.

\paragraph K07723.01 on \opstce\,005276332-01
This PC has a score of 0.262. It has four transits, one which was thrown out as a step-up by Marshall (Q10) and one that is on the edge of a gap (Q2). Likely this is an FP.

\paragraph K07716.01 on \opstce\,005097856-01
This PC has a score of 0.234. This is an extremely marginal PC but there is not obvious reason to fail it.  It is a 1.52 radius planet with an insolation flux of 0.39 around at 5700\,K G-dwarf star. The orbital period is 483\,d.

\paragraph K02853.02 on \opstce\,010387742-02
This FP has a score of 0.217 with 5 transit events.  One is a likely SPSD (Q5). Marshall flags Q11 and Q14 as offset only. Likely this is an FP.

These are not new to DR25.

\paragraph K07223.01 on \opstce\,009674320-01 
This candidate is was found first in the DR24 catalog and has 5 transits, 3 of which passed the Robovetter's metrics. It has a period of 317.091\,d and mes of 10.3247 and a score of 0.947. The 1.65 radii (-0.12 +0.27) planet circles a 5366\,K star with an insolation flux of 0.54 (-0.13 +0.29), making the planet a bit cooler than the Earth.

