%Section Completeness and Reliability

The completeness and reliabilty of the dispositions are evaluated using the simulated data sets,\injtce, \invtce, and \scrtce, described in \S\ref{s:simulated}  data sets. The Robovetter is mostly agnostic to the transit fit and stellar parameters and so measuring the completeness and reliability of the resulting catalog is best understood in terms of parameters that are similarly blind to that process.  Here we discuss the performance of the robovetter and the measured completeness and reliability of the catalog in terms of raw parameters such as MES and period, as well as derived parameters such as planet radius and insolation flux. 

\subsection{Completeness}
Figure 6 from Coughlin needs to be redone here.


\subsection{How good are the simulated data sets}
We should show some evidence that the fraction of false alarms in the simulated data sets matches those in the actual OBS data.  Can we show that the fraction of fails by LPP, individual transits and sig-pri/Fred are approximately the same between the two sets.  Or how different are they.  Do this for low mes,

\subsection{Effectiveness}

\subsection{Reliability}
Create a coarser Coughlin, Figure 6 for Reliability 


\subsection{Summarize}
We balanced the completeness and reliability

