%%List Acronyms Here
\subsection{Terms and Acronyms}
\label{abbrev}
We try to avoid unnecessary acronyms and abbreviations, but a few are required to efficiently discuss this catalog.  Here we itemize those terms and abbreviations that are specific to this paper and are used repeatedly. The list is short enough that we choose to group them by meaning instead of alphabetically. 

\begin{itemize}

\item[] \textbf{TCE}: Threshold Crossing Event. Periodic signals identified by the Kepler Pipeline. See Transit Planet Search in \citet{JenkinsKDPH}.
\item[] \textbf{\opstce}: Observed TCEs. TCEs found by searching the observed DR25 \Kepler\ data and reported in \citet{Twicken2016}.
\item[] \textbf{\injtce}: Injected TCE. TCEs found that match a known, injected transit signal \citep{Christiansen2017}.
\item[] \textbf{\invtce}: Inverted TCE. TCEs found when searching the inverted data set in order to simulate instrumental false alarms \citep{Coughlin2017a}.
\item[] \textbf{\scrtce}: Scrambled TCE. TCEs found when searching the scrambled data set in order to simulate instrumental false alarms \citep{Coughlin2017a}.
\item[] \textbf{TPS}: Transit Planet Search module. This module of the Kepler pipeline performs the search for planet candidates. This module identifies TCEs.
\item[] \textbf{DV}: Data Validation. Named after the module of the \Kepler\ Pipeline which characterizes the transits and outputs the \Kepler\ Pipeline's light curve detrending. DV is also used to refer to two sets of transit fits: original and supplemental \citet{JenkinsKDPH}.
\item[] \textbf{ALT}: Alternative. An alternative detrending, based on the methods of \citet{Garcia2010}, is utilized by several Robovetter metrics. A trapezoidal fit to the transit is performed on the ALT detrended light curves.
\item[] \textbf{MES}: Multiple Event Statistic. A statistic that measures the significance of the observed transits in the detrended, whitened light curve \citep{Jenkins2002a}.
\item[] \textbf{SPSD}: Sudden Pixel Sensitivity Dropout.  After a cosmic ray hit a pixel can suddenly lose sensitivity and gradually gain sensitivity over a few hours.
\item[] \textbf{KOI}: Kepler Object of Interest. Periodic, transit-like, events that were significant enough to warrant further review. A KOI is identified with a KOI number.
\item[] \textbf{PC}: Planet Candidate. A TCE that passed all of the Robovetter tests and metrics.  
\item[] \textbf{FP}: False Positive. A TCE that failed one or more of the Robovetter tests and metrics.
\item[] \textbf{MCMC}: Markov chain Monte Carlo. This refers to the transit fits that provide the planetary parameters and error bars for all KOIs \citep{Hoffman2017}.



\end{itemize}