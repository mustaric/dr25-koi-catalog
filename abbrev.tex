%%List Acronyms Here
\subsection{Terms and Abbreviations}
\label{abbrev}
We try to avoid unnecessary acronyms and abbreviations, but are a few are required to efficiently discuss this catalog.  Here we itemize those terms and abbreviations that are specific to this paper and are repeatedly used:

\begin{itemize}
%\item[] \textbf{KOI}: Kepler Object of Interest. A particular ephemeris on an object that was given a KOI number. The intent is to create KOIs out of all identified events that could possibly be planet candidates or eclipsing binaries.  
\item[] \textbf{TCE}: Threshold Crossing Event. Periodic signals identified by the Kepler Pipeline.
\item[] \textbf{\opstce}: Observed TCEs. TCEs found by searching the observed DR25 \Kepler\ data and reported in \citet{Twicken2016}.
\item[] \textbf{\injtce}: Injected TCE. TCEs found that match a known, injected transit signal.
\item[] \textbf{\invtce}: Inverted TCE. TCEs found when searching the inverted data set in order to simulate instrumental false alarms.
\item[] \textbf{\scrtce}: Scrambled TCE. TCEs found when searching the scrambled data set in order to simulate instrumental false alarms.
\item[] \textbf{KOI}: Kepler Object of Interest. Periodic, transit-like, events that were significant enough to warrant further review. 
\item[] \textbf{PC}: Planet Candidate. A TCE that passed all of the Robovetter tests and metrics.
\item[] \textbf{FP}: False Positive. A TCE that failed one or many of the Robovetter tests and metrics.
\item[] \textbf{MES}: Multiple Event Statistic. A statistic that measures the significance of the observed transits in the TPS detrended, whitened light curve \citep{Jenkins2002a}.
\item[] \textbf{DV}: Data Validation. The module of the \Kepler\ Pipeline which characterizes the transits. This module evaluates each TCE and provides a detrended light curves on which other metrics are based. The DV module also provided two sets of transit fits: original and supplemental.
\item[] \textbf{ALT}: Alternate detrending. An alternate detrending, based on the methods of \citet{Garcia2010}, which is utilized by several Robovetter metrics. A trapezoidal fit to the transit is performed on the light curves that results from this detrending.
\item[] \textbf{MCMC}: Markov chain Monte Carlo. This refers to the transit fits that are provided for all KOIs. See \S\ref{s:mcmc}.

\end{itemize}
