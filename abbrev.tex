%%List Acronyms Here
\subsection{Terms and Abbreviations}
\label{abbrev}
We try to avoid unnecessary acronyms and abbreviations, but are a few are required to efficiently discuss this catalog.  Here we itemize those terms and abbreviations that are repeatedly used in this paper:

\begin{itemize}
%\item[] \textbf{KOI}: Kepler Object of Interest. A particular ephemeris on an object that was given a KOI number. The intent is to create KOIs out of all identified events that could possibly be planet candidates or eclipsing binaries.  
\item[] \textbf{TCE}: Threshold Crossing Event identified by the Kepler Pipeline and reported in \citet{Twicken2016}.
\item[] \textbf{OBS-TCE}: Observed TCEs. TCEs found by searching the observed data, i.e. the DR25, Q1--Q17 data set.
\item[] \textbf{INJ-TCE}: Injected TCE. TCEs found that matches a known, injected transit signal.
\item[] \textbf{INV-TCE}: Inverted TCE. TCEs found when searching the inverted data set in order to simulate instrumental false alarms.
\item[] \textbf{SCR-TCE}: Scrambled TCE. TCEs found when searching the scrambled data set in order to simulate instrumental false alarms.
\item[] \textbf{KOI}: Kepler Object of Interest. Periodic events identified on a particular star that possibly have the shape of a transit or eclipse are added to the KOI table at NExScI. 
\item[] \textbf{PC}: Planet Candidate. A TCE that passed all of the Robovetter tests and metrics.
\item[] \textbf{FP}: False Positive. A TCE that failed one or many of the Robovetter tests and metrics.
\item[] \textbf{MES}: Multiple Event Statistic. A statistic that measures the significance of the observed transits in the TPS detrended, whitened light curve, see \citep{Jenkins2002a}.
\item[] \textbf{DV}: Data Validation. The module of the \Kepler\ pipeline which characterizes the transits. This module evaluates each TCE and provides a detrended light curves on which other metrics are based.
\item[] \textbf{ALT}: Alternate. An alternate detrending, based on the methods of \citet{Garcia2010}, are utilized by several metrics discussed in this paper.

\end{itemize}
