

\section{Conclusions}
\label{s:conclusions}

The DR25 KOI catalog has been characterized so that it can serve as the basis for occurrence rate studies of exoplanets with periods as long as 500 days. The detection efficiency of the entire search \citep{Burke2017b,Christiansen2017} and of the Robovetter vetting process \citep{Coughlin2017a} has been calculated by injecting planetary transits into the data and determining which types of planets are found and which are missed. For this DR25 KOI catalog, the vetting completeness has been balanced against the catalog reliability, i.e., how often false alarms are mistakenly classified as PCs. This is the first \Kepler{} exoplanet catalog to be characterized in this way, enabling occurrence rate measurements at the detection limit of the mission.  As a result, accurate measurements of the frequency of terrestrial-size planets at orbital periods of hundreds of days is possible.

The measurement of the reliability using the inverted and scrambled light curves is new to this KOI catalog. We measure how often noise is labeled as a planet candidate and combine that information with the number of false alarms coming from the \Kepler{} Pipeline. Some pure noise signals so closely mimic transiting signals that it is nearly impossible to remove them all. Because of this, it is absolutely imperative that those using this candidate catalog for occurrence rates consider this source of noise. For periods longer than $\approx$200 days and radii less than $\approx$4\,\Rearth, these noise events are often labelled as PC and thus the reliability of the catalog is near 50\%.  Astrophysical reliability is another concern that must be accounted for independently.  However, even once it is shown that another astrophysical scenario is unlikely \citep[as was done for the DR24 KOIs in][]{Morton2016}, the PCs in this catalog cannot be validated without first showing that the candidates have a sufficiently high false alarm reliability. 

We have shown several ways to identify high reliability or high completeness samples. Reliability is a strong function of the MES and the number of observed transits. Also, the FGK dwarf stars are known to be quieter than giant stars and in general the true transits can be more easily separated from the false alarms. We also provide the disposition score, a measure of how robustly a candidate has passed the Robovetter; this can be used to easily find the most reliable candidates. Those doing follow-up observations of KOIs may also use this disposition score to identify the candidates that will optimize ground-based follow-up observations.  

This search of the \Kepler{} data yielded 219 new PCs. Among those new candidates are two new candidates in multi-planet systems (KOI-82.06 and KOI-2926.05).  Also, the catalog contains ten new high-reliability, super-Earth size, habitable zone candidates.  Some of the most scrutinized signals in the DR25 KOI catalog will likely be those fifty small, temperate PCs in the eta-Earth sample defined in \S\ref{s:hz}.  These signals, along with their well characterized completeness and reliability, can be used to make an almost direct measurement on the occurrence rate of planets with the size and insolation flux as Earth, especially around GK dwarf stars.  While this catalog is an important step forward in measuring this number, it is important to remember a few potential biases inherent to this catalog. Namely, errors in the stellar parameters result in significant errors on the planetary sizes and orbital distances, and unaccounted for background stars make planet radii appear smaller than reality and impact the detection limit of the search for all stars.  Also, the Robovetter is not perfect --- completeness of the vetting procedures and the reliability of these signals (both astrophysical and false alarm) must be considered in any calculation.


Ultimately, characterizing this catalog was made possible because of the Robovetter (\S\ref{s:robovetter}) and the innovative metrics it uses to vet each TCE. It has improved the uniformity and accuracy of the vetting process and has allowed the entire process to be tested with known transits and known false positives. As a result, the Robovetter could be run many times, each time improving the vetting by changing thresholds or introducing new metrics. We adapted our vetting process as we learned about the data set, ensuring the highest reliability and completeness achievable in the time allowed.  The Robovetter metrics and logic may prove useful for future transit missions that will find an unprecedented abundance of signals that will require rapid candidate identification for ground-based follow-up, e.g., K2 \citep{Howell2014}, TESS \citep{Ricker2015}, and PLATO \citep{Rauer2016}.  


