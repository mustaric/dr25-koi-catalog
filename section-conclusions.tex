

\section{Conclusions}

The DR25 KOI catalog of exoplanet candidates has been characterized so that it can be used to measure the occurrence rate of long-period exoplanets. The detection efficiency of the entire search \citep{Burke2017b,Christiansen2017} and of the Robovetter vetting process has been performed by injecting planetary transits into the data, providing a complete measure the catalog completeness. That completeness has been balanced with a measurement of how often noise is mistaken for candidates, i.e. the catalog reliability. This is the first exoplanet catalog to be characterized in this way, enabling occurrence rate measurements at the detection limit of the mission.  With tools to understand the completeness and reliability of the catalog, accurate measurements of the frequency of terrestrial-sized planets in orbital periods longer than one-hundred days are possible.

The measurement of the planet candidate reliability using the inverted and scrambled light curves is new to this KOI catalog. We measure often noise is labeled as a planet candidate and combine it with the number of false alarms coming from the \Kepler\ pipeline. Some pure noise signals so closely mimic transiting signals that it is nearly impossible to remove them all. Because of this, it is absolutely imperative that those using the candidate catalog for occurrence rates, consider this source of noise. For periods longer than $\sim$200 days and radii less than $\sim$4\Rearth, false alarm candidates exist and the candidate reliability is near 50 per cent.  Astrophysical reliability is another concern that has not been included in this analysis.  However even once it is shown that another astrophysical scenario is unlikely \citep[as was done for the DR24 KOIs in ][]{Morton2017}, the candidate cannot be validated without first eliminating the false alarm reliability. 

We have shown several ways to identify a subset of the catalog that has a higher reliability, though a worse completeness. Reliability is a strong function of the MES, the number of observed transits and the transit duration. Also, the FGK dwarf stars are known to be quieter and in general the true transits can be separated from the known noise. We also provide the disposition score, a measure of when a candidate has barely passed the Robovetter, which can be used to find the highest reliability candidates, without removing the most interesting long period, low signal-to-noise candidates. For those doing follow-up observations of KOIs, may find this disposition score a convenient way to identified the best candidates for optimizing ground-based telescope time.

Ultimately, characterizing this catalog was made possible because of the Robovetter and the innovative metrics (\S\ref{s:robovetter}) it uses to vet each TCE. It has improved the uniformity and accuracy of the vetting process and has allowed the entire process to be tested with true transits and true noise signals. As a result the Robovetter could be run many times and the vetting could be improved by changing thresholds or introducing new metrics. As a result we could adapt our vetting process as we learned more about the data set and this ensures the highest reliability and completeness achievable in the time allowed.  The Robovetter metrics and logic may prove useful for future transit missions that will also find an abundance of signals and need to rapidly find the best ones to follow-up with observations from the ground, e.g. K2 \citep{Howell2014}, TESS \citep{Ricker2015}, and PLATO \citep{Rauer2016}. 


