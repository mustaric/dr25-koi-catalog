

\section{Conclusions}

The DR25 KOI catalog of exoplanet candidates has been carefully characterized so that it can be the input to the occurrence rate studies of exoplanets in periods as long as 500 days. The detection efficiency of the entire search \citep{Burke2017b,Christiansen2017} and of the Robovetter vetting process has been calculated by injecting planetary transits into the data and determining which types of planets are found and which are missed. For this DR25 KOI catalog, the vetting completeness has been balanced against how often noise is mistaken for candidates, i.e. the catalog reliability. This is the first \Kepler{} exoplanet catalog to be characterized in this way, enabling occurrence rate measurements at the detection limit of the mission.  As a result accurate measurements of the frequency of terrestrial-sized planets in orbital periods of hundreds of days is possible.

The measurement of the false alarm reliability using the inverted and scrambled light curves is new to this KOI catalog. We measure how often noise is labeled as a planet candidate and combine that information with the number of false alarms coming from the \Kepler{} Pipeline. Some pure noise signals so closely mimic transiting signals that it is nearly impossible to remove them all. Because of this, it is absolutely imperative that those using this candidate catalog for occurrence rates consider this source of noise. For periods longer than $\sim$200 days and radii less than $\sim$4\Rearth, noise events are often labelled as a PC and thus the reliability of the catalog is near 50\%.  Astrophysical reliability is another concern that must be accounted for independently.  However, even once it is shown that another astrophysical scenario is unlikely \citep[as was done for the DR24 KOIs in ][]{Morton2017}, the PCs in this catalog cannot be validated without first showing that the candidates have a sufficiently high false alarm reliability. 

We have shown several ways to identify a subset of the catalog that has a higher reliability, though a worse completeness. Reliability is a strong function of the MES, the number of observed transits and the transit duration. Also, the FGK dwarf stars are known to be quieter than giant stars and in general the true transits can be more easily separated from the known noise. We also provide the disposition score, a measure of how robustly a candidate has passed the Robovetter, which can be used to find the highest reliability candidates, without removing the most interesting long period, low signal-to-noise candidates. Those doing follow-up observations of KOIs may find this disposition score a convenient way to identify the best candidates for optimizing ground-based telescope time for follow-up.

Ultimately, characterizing this catalog was made possible because of the Robovetter and the innovative metrics (\S\ref{s:robovetter}) it uses to vet each TCE. It has improved the uniformity and accuracy of the vetting process and has allowed the entire process to be tested with true transits and true false positives. As a result the Robovetter could be run many times, each time improving the vetting by changing thresholds or introducing new metrics. We adapted our vetting process as we learned more about the data set, ensuring the highest reliability and completeness achievable in the time allowed.  The Robovetter metrics and logic may prove useful for future transit missions that will find an unprecedented abundance of signals that will require rapid identification of the candidates for ground-based follow-up, e.g. K2 \citep{Howell2014}, TESS \citep{Ricker2015}, and PLATO \citep{Rauer2016}. 


