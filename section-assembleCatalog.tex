%\section{The DR25 Catalog}
The KOI catalog contains all the \opstce s that the Robovetter found to have some chance of being transit-shaped. All of the DR25 KOIs are fit using a Monte Carlo approach. We describe here how we assemble the catalog using the information from the \opstce\ list, the Robovetter results and the transit fits. Finally we describe all the data that was delivered to the NASA Exoplanet Archive for public access.
\label{koisec}

\subsection{Dispositions}
The Robovetter gives every \opstce\ a disposition, a reason for the disposition and a score.  However, only those that are transit-like, i.e. have some possibility of being an exoplanet or binary system, are intended to be placed in the KOI catalog. For scheduling reasons, we created the majority of KOIs before we completed the Robovetter, so quite a few not-transit-like KOIs have been added to the KOI catalog. Using the final set of Robovetter dispositions, we made sure to include the following \opstce s in the KOI table: 1) those that are "transit-like", i.e. are not marked with the N-flag and 2) KOIs marked with the N-flag which have a score value larger than 0.1.  This last group were included to ensure that \opstce s that marginally failed one Robovetter metric were easily accessible via the KOI catalog. As in previous catalogs all \opstce{s} that match a previously identified KOI are included in the DR25 KOI table regardless of disposition.


\subsection{Federating to known KOIs}
All \opstce s that that we decided to include in the KOI catalog were either federated to known KOIs or given a new KOI number. Since KOIs have been identified before, federating the known KOIs to the TCE list is a necessary step to ensure that we do not create new KOIs out of events previously identified by the \Kepler\ pipeline.  The process has not changed from the  DR24 KOI catalog \citep{Coughlin2016}, so we simply summarize it here.  For each \opstce\ we use the ephemeris to determine what fraction of in-transit cadences overlap between all known KOIs on the same KIC.  Those with a significant overlap are considered federated.  Also, those that are found at double or half the period are also considered matches.  
%[CHRIS CAN I be more precise here?].

In some cases our automated tools want to create a new KOI on a system where one of the KOIs was not federated.  In these cases we inspect the new system by hand and ensure that it truly warrants a new KOI number. If it is a new event we create a new KOI. If not, we ban the event from the KOI list.  For instance, events that are caused by an image artifact known as cross-talk \citet{Coughlin2014a} can cause short-period transit events to appear in only one quarter each year. As a result the \Kepler\ pipeline finds several one-year period events for an astrophysical event that is truly closer to a few days.  In these cases we federate the one found that most closely matches the known KOI and we ban any other \opstce\ from creating a new KOI around this star. In Table\,\ref{t:banned} we report the entire list of \opstce{s} that were not made into KOIs despite the federation telling us that one was appropriate.

\begin{deluxetable}{c}
\tablecolumns{1}
\tablewidth{\linewidth}
\tablecaption{\opstce{s} Banned from Becoming KOIs (\S\ref{s:federation}) \label{t:banned}}
\tablehead{
\colhead{TCE-ID}\\
\colhead{(KIC-PN)}
}
\startdata
003340070-04\\
003958301-01\\
005114623-01\\
005125196-01\\
005125196-02\\
005125196-04\\
005446285-03\\
006677841-04\\
006964043-01\\
006964043-05\\
007024511-01\\
008009496-01\\
008956706-01\\
008956706-06\\
009032900-01\\
009301564-01\\
010223616-01\\
012459725-01\\
012644769-03\\
\enddata

\end{deluxetable}


% Original List Before Jeff Touched It
% \startdata
%  003340070-04 \\
%  003958301-01 \\
%  005114623-01 \\
%  005125196-01 \\
%  005125196-02 \\
%  005125196-04 \\
%  005446285-03 \\
%  006677841-03 \\
%  006677841-04 \\
%  006964043-01 \\
%  006964043-05 \\
%  007024511-01 \\
%  008009496-01 \\
%  008956706-01 \\
%  008956706-06 \\
%  009032900-01 \\
%  009301564-01 \\
%  010223616-01 \\
%  011661803-02 \\
%  012459725-01 \\
%  \enddata
