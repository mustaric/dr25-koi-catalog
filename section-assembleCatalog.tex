%\section{The DR25 Catalog}
The KOI catalog contains all the \opstce s that the Robovetter found to have some chance of being transit-shaped, i.e., astrophysically transiting or eclipsing systems. All of the DR25 KOIs are fit with a transit model and uncertainties for each model parameter are calculated with a MCMC algorithm.  We describe here how we decide which \opstce{s} become KOIs, how we match the \opstce{s} with previously known KOIs, and how the transit fits are performed. The KOI catalog is available at the NASA Exoplanet Archive as the Q1-Q17 DR25 KOI Table\footnote{\url{https://exoplanetarchive.ipac.caltech.edu/cgi-bin/TblView/nph-tblView?app=ExoTbls\&config=q1\_q17\_dr25\_koi}}.
\label{koisec}

\subsection{Creating KOIs}
The Robovetter gives every \opstce\ a disposition, a reason for the disposition, and a score.  However, only those that are transit-like, i.e., have some possibility of being a transiting exoplanet or eclipsing binary system, are intended to be placed in the KOI catalog. For scheduling reasons, we created the majority of KOIs before we completed the Robovetter, so quite a few not transit-like KOIs have been included in the KOI catalog. Using the final set of Robovetter dispositions, we made sure to include the following \opstce s in the KOI table: 1) those that are ``transit-like'', i.e., are not marked with the NT-flag, and 2) KOIs that are not transit-like FPs which have a score value larger than 0.1.  This last group were included to ensure that \opstce s that marginally failed one Robovetter metric were easily accessible via the KOI catalog and given full transit fits with MCMC error bars. As in previous catalogs, all DR25 \opstce{s} that federate (\S\ref{s:federation}) to a previously identified KOI are included in the DR25 KOI table even if the Robovetter set the disposition to a not transit-like FP. All previous KOIs that were not found by the DR25 \Kepler{} Pipeline (i.e., did not trigger a DR25 \opstce{}) are not included in the DR25 KOI table at the Exoplanet Archive.


\subsection{Federating to known KOIs}
\label{s:federation}
All \opstce s that were included in the KOI catalog were either federated to known KOIs or given a new KOI number. Since KOIs have been identified before, federating the known KOIs to the TCE list is a necessary step to ensure that we do not create new KOIs out of events previously identified by the \Kepler\ pipeline.  The process has not changed from the DR24 KOI catalog \citep[see \S4.2 of][]{Coughlin2016}, so we simply summarize it here.  For each \opstce\ we use the ephemeris to determine what fraction of in-transit cadences overlap with all known KOIs on the same star.  Those with significant overlap are considered federated.  Also, those that are found at double or half the period are also considered matches (244 KOIs in total).  
%[CHRIS CAN I be more precise here?].

In some cases our automated tools want to create a new KOI in a system where one of the other previously known KOIs in the system did not federate to a DR25 TCE.  In these cases we inspect the new system by hand and ensure that a new KOI number is truly warranted. If it is, we create a new KOI. If not, we ban the event from the KOI list.  For instance, events that are caused by video cross-talk \citep{KIH} can cause short-period transit events to appear in only one quarter each year. As a result, the \Kepler\ Pipeline finds several one-year period events for an astrophysical event that is truly closer to a few days.  In these cases we federate the one found that most closely matches the known KOI and we ban any other \opstce{s} from creating a new KOI around this star. In Table~\ref{t:banned} we report the entire list of \opstce{s} that were not made into KOIs despite being dispositioned as transit-like (or not transit-like with a disposition score $\ge$~0.1) and the automated federation telling us that one was appropriate. To identify the TCEs we specify the Kepler Input Catalog number and the planet number given by the \Kepler{} Pipeline \citep{Twicken2016}.

\begin{deluxetable}{l}
\tablecolumns{1}
\tablewidth{\linewidth}
\tablecaption{Anomalous transit-like \opstce{s} not turned into KOIs \label{t:banned}}
\tablehead{
\colhead{TCE-ID}\\
\colhead{(KIC-pn)}
}
\startdata
 003340070-04 \\
 003958301-01 \\
 005114623-01 \\
 005125196-01 \\
 005125196-02 \\
 005125196-04 \\
 005446285-03 \\
 006677841-03 \\
 006677841-04 \\
 006964043-01 \\
 006964043-05 \\
 007024511-01 \\
 008009496-01 \\
 008956706-01 \\
 008956706-06 \\
 009032900-01 \\
 009301564-01 \\
 010223616-01 \\
 011661803-02 \\
 012459725-01 \\

\end{deluxetable}

It is worth pointing out that the banned TCEs is the one pseudo-manual step that is not repeated for all the simulated TCEs.  However, these banned TCEs effectively disappear when doing statistics on the catalog (i.e., these TCEs do not count as either a PC or an FP), are not present in the simulated data sets, nor are we likely to remove good PCs from our sample this way. Most banned TCEs are caused by either a short-period binary whose flux is contaminating our target star (at varying depths through mechanisms such as video cross-talk or reflection), or are strong TTV systems (see \S\ref{s:mcmc}). In both cases, the Pipeline finds several TCEs at various periods, but only one astrophysical system causes the signal.  By banning these \opstces, we are removing duplicates from the KOI catalog and improving the completeness and reliability statistics reported in \S\ref{s:candr}.
