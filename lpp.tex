%% Text describing LPP and how it works.

We implemented the LPP transit-like metric described by \citet{Thompson2015b} to separate those TCEs that show a transit-like shape from those that do not. This technique bins the folded light curve and then uses a machine learning technique that uses a dimensionality reduction algorithm called Locality Preserving Projections (LPP) to group transits of a similar shape.  In attempting to find a metric that best separates the known candidates from the false positives, we deviated from the method described \citet{Thompson2015b} in a few key ways.  This retraining was directed at solving the problem that the metric was highly dependent on both period and MES. The threshold that divides the transit-like from the not-transit-like events increases at periods less than $\sim$ 10??\,days. This is caused by the fact these TCEs transit for a large fraction of the orbital period and because transit injection does not include a large number of this type of event.  The trend with MES is difficut to overcome. It is routed in the fact that when the binned light curve has a lower signal to noise, it is less likely for two light curves to be identical to each other, creating more scatter in the final metric.


More specifically we made the following changes when implementing the LPP transit-metric. First, we trained the LPP algorithm using the pixel-level transit injections plus those planet candidates from DR24 that were found as TCEs in DR25.  Transit injection had very few short period injections, which can look distinctly different, even in the highly controlled binned space used by LPP. Since this population defines what a transit-like event looks like, it is important to have a wide variety of known transits, thus we included the DR24 candidates in the training sample.  Second, we changed how the data were binned. To remove some of the MES dependence TCEs with lower MES are given wider bins for those near the transit center. Also, we found that we got better performance by having a total ??97 bins (x for wide and y for central bins) and reducing them to 20?? dimensions with LPP.  Third, we normalized the LPP values by the 75 percentile of the N?? training set TCEs that are closest in period.  In this way we remove the period dependence on the LPP metric.  See Figure~{fig:lpp}.

The matlab code used to calculate this metric is available at the following repository:
