
%One of the goals of NASA's astrophysics missions is to understand the Earth's place in the Universe.
\Kepler{'s} mission to measure the frequency of Earth-sized planets in the Galaxy is an important step towards understanding the Earth's place in the Universe.  Launched in 2009, \Kepler{} stared almost continuously at a single field for four years (or 17, $\approx$90~day quarters), recording the brightness of $\approx$180,000 stars every 29.4 minutes\footnote{Kepler only observed $\approx$150,000 stars at a time}. \Kepler\ detected transiting planets by observing the periodic decrease in the observed brightness of a star when an orbiting planet crossed the line of sight from the telescope to the star. \Kepler{'s} observations concluded in 2014 when it lost a second of four reaction wheels required to maintain the stable pointing.  From the ashes of Kepler rose the K2 mission which continues to find exoplanets in addition to a whole host of astrophysics not enabled by the original \Kepler{} mission \citep{Howell2014}.


\Kepler\ data revolutionized exoplanet discoveries. Prior to \Kepler{,} exoplanets were primarily discovered by radial velocity methods \citep[e.g.][]{Mayor1995}, which largely resulted in the detection of Neptune to Jupiter mass planets in orbital periods of days to months. The high precision photometry and the long baseline of \Kepler{} data, extended the landscape of known exoplanets to moon-sized planets and terrestrial size planets in orbits as long as a year. To highlight a few examples, \citet{Barclay2013} found evidence for a moon-sized terrestrial planet in a 13.3 day period orbit, and \citet{Quintana2014} found evidence of an exoplanet in the habitable zone of the M dwarf Kepler-186. \Kepler\ data has also found hundreds of multi-planet systems, including the Kepler-11 system which hosts six planets all with periods less than 120 days \citep{Lissauer2011}. Most surprisingly, exoplanets have even been found orbiting binary stars, e.g. Kepler-16 \citep{Doyle2011}.

Other authors have taken advantage of the long time-base, near continuous data set of thousands of stars to advance our understanding of stellar physics through the use of asteroseismology. Of particular interest to this catalog is the improvements in our measurements of stellar radius \citep[e.g.][]{Huber2014a,Mathur2017} which is an important source of error when calculating planetary radii. Stellar magnetic activity was mapped enabling accurate measurements of stellar rotation \citep[e.g.][]{Aigrain2015,McQuillan2014}. Studying stars in clusters enabled \citet{Meibom2011} to map out the evolution of stellar rotation as a star ages. \Kepler{} also produced light curves of thousands of binary stars \citep{Prsa2011,Kirk2016} including unusual binary systems, such as the eccentric Heartbeat stars \citep{Welsh2011,Thompson2012} that have opened the doors to understanding the impact of tidal decay on stellar evolution.

The wealth of astrophysics, and the size of the \Kepler{} community, was in part due to the rapid release of \Kepler{} data to the NASA Archives: NExScI \citep[NASA Exoplanet Science Institute, ][]{Akeson2017} and MAST (Mikulski Archives for Space Telescope). The \Kepler{} mission releases data from every step of the processing, including unprocessed images, detrended light curves \citep{Stumpe2014}, lists of planet candidates, as well as ancillary data such as pixel response functions \citep{Bryson2010b}. The light curves used for the planet search discussed here are all part of Data Release 25 (DR25) and are described in detail in \citet{DRN25} and \citet{KDCH}.

The mission cataloged the results of many of its searches for transit signals in the data. The results of both the original search for periodic signals (known as the TCEs or Threshold Crossing Events) and the well-vetted KOI Table (Kepler Objects of Interest) are made available for the community.  Since the first candidate catalog by \citet{Borucki2011a}, each search has been on a progressively longer data set \citep{Batalha2013, Burke2014, Rowe2015cat, Mullally2015cat}, culminating in the DR24, Q1--Q17 KOI catalog \citet{Coughlin2016} and TCE Table \citep{Seader2015} respectively.  The results of the final search based on DR25 is presented here. While it does not include more data, it is the first catalog produced where all the data quarters have been processed by the same software pipeline \citep[version 9.3,][]{JenkinsKDPH}. The \Kepler{} Pipeline has undergone successive improvements since launch as the data characteristics have become better understood.

Eclipsing binaries systems often mimic the signal of a transit and must be weeded out through either additional confirming data \citep[such as radial velocities, e.g.][]{Marcy2014}, or statistical analysis. Even with rigorous vetting, most planet candidates in the KOI catalogs can not be directly confirmed as planetary. The stars are too dim and the planets are too small to be able to measure a radial velocity signature for the planet.   Statistical methods study the likelihood that the observed transit could be caused by other astrophysical scenarios and have succeeded in validating thousands of \Kepler{} planets \citep[e.g.][]{Morton2016, Torres2015,Lissauer2014, Rowe2014}.  

\citet{Mullally2015cat} first noted an additional source of false positives; that of the instrument itself. In that catalog (and again in this one), the majority of long-period, low SNR TCEs are ascribed to instrumental effects incompletely removed from the data before the TCE search. \Kepler\ has a variety of short timescale, non-Gaussian noise sources, from focus changes due to the changing solar illumination angle, signals imprinted on the data by the detector electronics, to the changing pixel sensitivity after the impact of a high energy particle (known as a sudden pixel sensitivity drop-out, SPSD). This catalog invested considerable effort in identifying such ``false alarms'', and for the first time we include a measurement of our success in doing so.

Because of the abundance of planet candidates, \Kepler{} catalogs have been used to understand the frequency of different types of planets in the Galaxy. \citet{Dressing2013,Dressing2015}, confined their analysis to M dwarfs and orbital periods less than 50\,d and determined that multi-planet systems are common around low mass stars.  Therefore planets are more common than stars in the Galaxy (due to the fact that low mass stars are the most common stellar type). \citet{Fulton2017}, using improved measurements of the stellar properties, looked at small planets in periods of less than 100\,d and showed that there is a dearth of planets near 1.75\re. Less is known for planets in longer period orbits. \Kepler{} has yet to produce a definitive estimate for the frequency of small planets in the Habitable Zone of sun-like stars \citep[see e.g.][]{ForemanMackey16, Petigura2013b, Burke2015}. \citet{Burke2015} used the Q1--Q16 \citep{Mullally2015cat} KOI catalog and looked at G and K stars and concluded that between 2-25\% of solar-type stars hosts planets with radii between 0.75--1.25\re and orbital periods within 25\% that of the Earth. The dominant uncertainty in their calculation was systematic; a better understanding of the properties of both the stars and the catalog generation process will narrow the uncertainty in that result. 

 

\subsection{Motivation for the DR25 catalog}

The DR25 KOI catalog was designed and developed to support rigorous occurrence rate studies. To do that well, we not only identify the exoplanet transit signals in the data but also measure the reliability (the fraction of candidates that are likely due to astrophysical signals) and the completeness of our catalog (the fraction of candidates detected).

The measurement of the catalog completeness has been split into two parts: the completeness of the TCE list (the transit search) and the completeness of the KOI catalog (the vetting of the TCEs). The completeness of the \Kepler\ Pipeline and its search for transits has been studied by injecting transit signals into the pixels and examining what fraction are found by the Pipeline \citep{Christiansen2017, Christiansen2015b,Christiansen2013a}. \citet{Burke2015} applied the resulting detection efficiency contours to the 50--300\,d period planet candidates in the Q1--Q16 KOI catalog \citep{Mullally2015cat} in order to measure the occurrence rates of small planets. However, this study was not able to account for those planets identified by the \Kepler{} Pipeline but thrown-out by the vetting process. Along with the DR25 KOI catalog, we provide a measure of the completeness of the vetting tools. 

While the reliability of \Kepler\ catalogs against astrophysical false positives is mostly understood \citep[see e.g.][]{Morton2016}, the reliability against false alarms (a term used in this paper to indicate TCEs caused by stellar variability or instrumental noise) has not previously been measured.  \Kepler\ light curves contain variability that is not due to planet transits or eclipsing binaries.  Instrumental noise, poor detrending, or stellar variability can conspire to make a signal that looks similar to a planet transit. When examining the smallest exoplanets in the longest orbital periods, \citet{Burke2015} demonstrated the importance of understanding the reliability of the catalog, showing that the occurrence of small, earth-like period planets around G stars changed by a factor of $\approx$10 depending on the instrumental reliability of a few planet candidates.  In this catalog we measure the reliability of the reported planet candidates against this instrumental and stellar noise.  

Completeness is measured by vetting thousands of injected transits found by the \Kepler Pipeline. Reliability is measured by vetting signals found in scrambled and inverted \Kepler\ light curves and counting the number of ``false-alarm" planet candidates that are created in the vetting process. This requirement to vet both the real and simulated TCEs in a reproducible and consistent manner caused us to use entirely automated methods to vet the TCEs.  

The automatic vetting was introduced in the Q1--Q16 KOI catalog \citep{Mullally2015cat} with the Centroid Robovetter and was then extended to all aspects of the vetting process for the DR24 KOI catalog \citep{Coughlin2016}. Because of this automation, the DR24 catalog was the first with a measure of completeness which extended to all parts of the search, from pixels to planets.  Now, with the DR25 KOI catalog and simulated false alarms, we also provide a measure of how effective the vetting techniques are at identifying noise signals and translate that into a measure of the catalog reliability. As a result, the DR25 KOI catalog is the first to balance the completeness and the reliability instead of erring on the side of high completeness. 

\subsection{Summary and Outline of the Paper}

The DR25 KOI catalog is a well-vetted list of planet candidates found in the DR25 \Kepler\ light curves with a measure of the catalog completeness and reliability. In the brief outline that follows we highlight how the catalog was assembled, how we measure the completeness and reliability, and emphasize those aspects of the process that are different from the DR24 KOI catalog \citep{Coughlin2016}.

In \S\ref{s:tces} we describe the observed TCEs, \opstce s, which are the periodic signals found in the actual \Kepler\ light curves. For reference, we also compare them to the DR24 TCEs. To create the simulated data sets necessary to measure the vetting completeness and the catalog reliability, we ran the \Kepler\ Pipeline on light curves which either contained injected transits, were inverted, or were scrambled. This creates \injtce s, \invtce s, and \scrtce s, respectively (see \S\ref{s:simulated}).  

We then created and tuned a Robovetter to vet the entire set of TCEs. \S\ref{s:robovetter} describes the metrics and the logic used to disposition TCEs into categories of planet candidates (PCs) or false positives (FPs).  Because the DR25 \opstce\ population was significantly different than the DR24 \opstce{s}, we developed new metrics to separate the PCs from the FPs (see \S\ref{s:metrics}.) Several new metrics examine the individual transits for evidence of being due to instrumental noise, see \S\ref{s:indivtrans}. As in the DR24~KOI catalog we group false positives into four categories (\S\ref{s:majorflags}) and provide false positive flags (\S\ref{s:minorflags}) to indicate why the Robovetter decided to pass or fail a TCE.  New to this catalog is the addition of a disposition score, see \S\ref{s:scores}, which gives users a measure of our confidence in the KOI disposition. Finally, we perform a full Markov chain Monte Carlo (MCMC) fit to each KOI, see \S\ref{s:mcmc}, in order to provide improved individual planet parameters and reliable error bars.


Unlike previous catalogs, this DR25 KOI catalog is no longer based on the philosophy of ``innocent until proven guilty"; we accept certain amounts of collateral damage in order to achieve a catalog that is uniformly vetted and has acceptable levels of both completeness and reliability, especially for the long period and low signal-to-noise PCs. In \S\ref{s:optimize} we discuss how we tuned the Robovetter using the simulated TCEs as populations of true planet candidates and true false alarms. We provide the Robovetter and all the Robovetter metrics for all of the TCEs (\opstce, \injtce, \invtce, and \scrtce) to enable users to create a catalog tuned for other regions of parameter space if their scientific goals require it. 

We then proceed to discuss the performance of the vetting and the properties of the catalog. We examine the vetting completeness and catalog reliability using the \injtce, \invtce and \scrtce\ sets in \S\ref{s:candr}. We show that both decrease significantly with decreasing number of transits and decreasing signal-to-noise.  We then discuss how one may use the disposition scores to identify the highest quality candidates, especially at long periods in \S\ref{s:crscores}.  We conclude that not all declared planet candidates in our catalog are truly planet candidates, but we can measure what fraction are caused by noise. Because of the interest in terrestrial, temperate planets, we examine the high quality, small candidates in the habitable zone in \S\ref{s:hz}. Finally, in \S\ref{s:occurates}, we give an overview of how this catalog can be combined with information in other \Kepler{} products to measure accurate occurrence rates of exoplanets.






%The \Kepler\ space telescope was launched in 2009 into an earth trailing orbit with the goal of finding and characterizing the frequency planets around other stars \citep{Koch2010,Lissauer2014}. It accomplished this goal by staring at the same field of the sky for four years (or 17, $\approx$90~day quarters) and collecting a measurement of the brightness of 180,000 stars every 29.4 minutes.  In this way it could look for evidence of transiting planets.  It finished operations in 2014 because it lost it's second reaction wheel and could no longer maintain stable pointing. From the ashes of Kepler rose the K2 mission\citet{Howell2014} which continues to find exoplanets in addition to a whole host of astrophysics not enabled by the original \Kepler{} mission \citep{}. 
%The \Kepler{} telescope observed the brightness of $\approx$180,000 stars for four years looking for evidence of a planetary transit.

%During \Kepler{'s} operations it revolutionized exoplanet discoveries. Prior to \Kepler{,} the known exoplanets were primarily discovered by periodic shifts in the radial velocity of the star \citep{}[INSERT large radial velocity studies]. These exoplanets were primarily in orbits of a few days and had masses similar to Jupiter's.   Because of the high precision photometry and the long baseline of the \Kepler\ photometry, the landscape of known exoplanets expanded to moon-sized planets and planets in orbits as long as a year. To highlight a few examples, \citep{Barclay2013} found evidence for a moon-sized terrestrial planet in a 13.3 day period orbit. \Kepler\ found evidence of exoplanets in the habitable zone of their host star, e.g. Kepler-186f \citep{Quintana2014}. The Kepler data has also shown how planets are commonly found in exoplanetary systems. These systems can be compact like the system around Kepler-11 \citep{Lissauer2007}[Check citation], where there are six planets all with periods less than XX days. Also, exoplanets have been found orbiting binary stars, e.g. Kepler-16\citep{Doyle2011}[OTHER CITATIONS].
%and Kepler-452b \citep{Jenkins2015}

%Because of the long continuous data set, \Kepler{} data advanced our understanding of stars. With the use of asteroseismology, \citet{}[CITATION] measured the radius and age of XX stars to unparalleled precision. The magnetic activity was mapped enabling acurate measurements of stellar rotation. By observing stars in clusters the evolution of stellar rotation as a star ages was mapped out (gyrochronology). Finally, \Kepler{} obtained light curve of more than XX binary stars \citep{Prsa2011,Kirk2016}[CITATIONS] (Binary star results). Unusual binary systems, such as the eccentric Heartbeat stars \citet{Welsh2011,Thompson2012} have opened the doors to understanding tidal evolution.

%The variety of astrophysics and the size of the \Kepler{} community was in part due to the rapid release of \Kepler{} data to the NASA Archives: NExScI (NASA Exoplanet Science Institute) and MAST (Mikulski Archives for Space Telescope).  Besides calibrated images and detrended light curves, the mission has cataloged the results of its searches for transit signals in the data. Both the automated, raw search for periodic signals (known as the TCE (Threshold Crossing Event) Table) and the well-vetted KOI Table (Kepler Objects of Interest) are available for the community to search.  Since the first KOI table by \citet{Borucki2011a}, each search has been on a progressively longer data set, culmination in the Q1--Q17 KOI catalog and TCE Table by \citet{Coughlin2016} and \citet{Seader2015} respectively.  The results of the final search based on Data Release 25 (DR25) is presented here. While it does not include more data, it is the first time all the data quarters have been processed with the same software pipeline, which has been improved since the early exoplanet searches by reducing the noise and flagging and correcting systematic noise [citations...specifics??].

%This last search of the data has been done with the goal of enabling the mission to regarding the question of the frequency of small planets in the 
 


%Despite all of \keplers{} successes it has not produced a definitive number for the frequency of small planets in the Habitable Zone of sun-like stars. \citet{Dressing2013,Dressing2015}, lots of occurrence rates papers.   Burke shows us that 
%The impact of \Kepler\ on our understanding of exoplanets comes not only from these individual systems, but from the plethora of planets of every type that it has found. Understanding how common different types of exoplanets are around different types of stars will be one of its lasting legacy. \Kepler\ has shown us that most stars have exoplanets and that super-Earth and small planets are more common than their larger cousins \citet{Burke2015}.  
%Note, this is separate from the reliability against astrophysical false positives, such as background eclipsing binaries, which has been well studied by \citet{Morton2016} and \citet{Torres2012} by comparing the likelihood of the transiting planet model against other astrophysical scenarios.
%And for each of these unique systems, there are a plethora of exoplanets This plethora of planets \Kepler\ has shown that small planets are common 

%--Describe KOI Tables, what is a TCE.  Quarters, Seasons.
%This telescope began collecting photometry of $\approx$ 180,000 stars in 2010 at at 29.4 minute cadence in 90 day segments, called quarters. After each quarterly download the telescope would rotate to keep solar panels pointed at the sun

%Mention something about balancing completeness and reliabiltiy over innocent until proven guilty. [ Kelsey]

%Since \Kepler's launch in 2009 it has succeeded in identifying X,XXX planet candidates \citep{Borucki2010a,Coughlin2016}. 

%X,XXX of these exoplanets have gone on to be ... or validated planets \citep[see most recently][]{Morton2016}.  % \citep[e.g.][]{FOP?}

%From Jack
%The primary goal of the Kepler Mission was to determine occurrence rates of planets as functions of planetary size, orbital period and stellar type \citep{Boruck2004}.  The purpose of this catalog is to provide a listing of planet candidates that have been found and vetted in a well-defined and reproducible manner for the scientific community to use as input for studies of planet occurrence rates.  As such, the data were processed and vetted in a fully automated manner, using algorithms based on years of experience gained from a combination of code development and hands-on work with individual TCEs.  

%[This part is kind wrong...but the sentiment needs a place] It differs from the previous Kepler catalog in two key respects: previously-identified planet candidates that did not appear as TCEs in DR25 do not appear in this catalog, and the vetting of individual candidates did not include the hands-on treatment that was a feature of (all previous? or the first six or whatever) Kepler planet candidate catalogs.  This optimizes involves decisions that compromise the evaluation of some particular planet candidates, as detailed further in Section XX.
%As such, the data were processed and vetted in a fully automated manner, using algorithms based on years of experience gained from vetting individual signals by hand.