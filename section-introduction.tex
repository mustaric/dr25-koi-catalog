
%One of the goals of NASA's astrophysics missions is to understand the Earth's place in the Universe.
\Kepler{'s} mission to measure the frequency of Earth-sized planets in the Galaxy is an important step towards understanding the Earth's place in the Universe.  Launched in 2009, the \Kepler{} Mission \citep{Koch2010,Borucki2016} stared almost continuously at a single field for four years (or 17, $\approx$90~day quarters), recording the brightness of $\approx$200,000 stars ($\approx$160,000 stars at a time) at a cadence of 29.4 minutes over the course of the mission. \Kepler{} detected transiting planets by observing the periodic decrease in the observed brightness of a star when an orbiting planet crossed the line of sight from the telescope to the star. \Kepler{'s} observations concluded in 2013 when it lost a second of four reaction wheels, three of which were required to maintain the stable pointing.  From the ashes of \Kepler{} rose the \Ktwo{} mission which continues to find exoplanets in addition to a whole host of astrophysics not enabled by the original \Kepler{} mission \citep{Howell2014,VanCleve2016K2}.


\Kepler{} data revolutionized exoplanet discoveries. Prior to \Kepler{,} exoplanets were primarily discovered by radial velocity methods \citep[e.g.][]{Mayor1995}, which largely resulted in the detection of Neptune- to Jupiter-mass planets in orbital periods of days to months. The high precision photometry and the long baseline of the \Kepler{} data extended the landscape of known exoplanets to moon-sized planets and terrestrial-sized planets in orbits as long as a year. To highlight a few examples, \citet{Barclay2013} found evidence for a moon-sized terrestrial planet in a 13.3 day period orbit, \citet{Quintana2014} found evidence of an Earth-sized exoplanet in the habitable zone of the M dwarf Kepler-186, and \citet{Jenkins2015} validated a super-Earth in the habitable zone of a G-dwarf star. Additionally, for several massive planets \Kepler{} data has enabled measurements of planetary mass and atmospheric properties by using the photometric variability along the entire orbit \citep{Shporer2011,Mazeh2012,Shporer2017}. \Kepler\ data has also revealed hundreds of compact, co-planar multi-planet systems, e.g., the six planets around Kepler-11 \citep{Lissauer2011}, which collectively have told us a great deal about the architecture of planetary systems \citep{Lissauer2011b,Fabrycky2014}.  Finally, exoplanets have been found orbiting binary stars, e.g., Kepler-16~(AB)~b \citep{Doyle2011}.

Other authors have taken advantage of the long time-base, near continuous data set of hundreds of 206,150\footnote{This tally only includes the targeted stars and not those observed by ``accident'' in the larger apertures.} stars to advance our understanding of stellar physics through the use of asteroseismology. Of particular interest to this catalog is the improvement in the determination of stellar radius \citep[e.g.,][]{Huber2014a,Mathur2017ApJS} which can be one of the most important sources of error when calculating planetary radii. \Kepler{} data was also used to track the progress of spots created from magnetic activity which enabled the determination of stellar rotation \citep[e.g.][]{Aigrain2015,Garcia2014,McQuillan2014}. Studying stars in clusters enabled \citet{Meibom2011} to map out the evolution of stellar rotation as stars age. \Kepler{} also produced light curves of 2876\footnote{This represents the number reported in the Kepler Binary Catalog, \url{http://keplerebs.villanova.edu}, in August 2017.} eclipsing binary stars \citep{Prsa2011,Kirk2016} including unusual binary systems, such as the eccentric Heartbeat stars \citep{Welsh2011,Thompson2012,Shporer2016hb} that have opened the doors to understanding the impact of tidal forces on stellar pulsations and evolution \citep[e.g.,][]{Hambleton2017,Fuller2017}.

The wealth of astrophysics, and the size of the \Kepler{} community, is in part due to the rapid release of \Kepler{} data to the NASA Archives: Exoplanet Archive \citep[][]{Akeson2013} and MAST (Mikulski Archives for Space Telescopes). The \Kepler{} mission released data from every step of the processing, including unprocessed images, systematic-error-corrected light curves \citep{Stumpe2014}, lists of planet candidates, as well as ancillary data such as pixel response functions \citep{Bryson2010b}. The light curves used for the planet search discussed here are all part of Data Release 25 (DR25) and are described in detail in \citet{DRN25} and \citet{KDCH}.

The mission cataloged the results of many of its searches for periodic transit signals in the data.  The results of both the original searches for periodic signals (known as the TCEs or Threshold Crossing Events) and the well-vetted KOIs (Kepler Objects of Interest) are made available for the community.  The combined list of \Kepler{'s} planet candidates, containing information from the most recent vetting, can be found in the cumulative KOI table. This table is actually the combination of individual KOI tables that resulted from searches with progressively longer baselines \citep{Borucki2011a, Batalha2013, Burke2014, Rowe2015cat, Mullally2015cat,Coughlin2016}. We present here the results of the final search, based on DR25. While the search does not include new observations, it was performed using an improved version of the \Kepler{} Pipeline \citep[version 9.3,][]{Jenkins2017}. For a high-level summary of the changes to the Pipeline, see the DR25 data release notes \citep{DRN25}. The \Kepler{} Pipeline has undergone successive improvements since launch as the data characteristics have become better understood.

{\color{blue}
The photometric ``noise'' at time scales of the transit is what limits \Kepler{} from finding small terrestrial-sized planets. Investigations of the noise properties of \Kepler{} exoplanet hosts by \citet{Howell2016} showed that those exoplanets with the smallest radii and shallowest transits ($\leq 200$ parts per million) are only found around those stars that are photometrically quiet. As a result, the search for the truly Earth-sized planets are limited to a small subset of \Kepler{'s} stellar sample.  Analyses by \citet{Gilliland2011,Gilliland2015} show that the primary source of the observed noise was indeed inherent to the stars, with a smaller amount coming from imperfections of the instruments and software. Unfortunately, the typical noise-level for 12$^{th}$ magnitude solar-type stars is closer to 30\,ppm \citep{Gilliland2015} than the 20\,ppm expected from \citet{Jenkins2002a}, causing \Kepler{} to need a longer baseline to find a significant number of Earth-like planets around Sun-like stars.   Ultimately, this higher noise level impacts \Kepler{'s} planet yield. And, because different stars have different levels of noise, the transit depth searched for each star varies across the sample. This bias must be accounted for when calculating occurrence rates and is explored in-depth, for this run of the \Kepler{} Pipeline, by the transit injection and recovery studies of \citet[][]{Burke2017b,Burke2017a,Christiansen2017}.
}
 

%%This might be a good place for a paragraph regarding follow-up observing.
{\color{blue}
To confirm the validity and further characterize the identified planet candidates, the \Kepler{} mission benefited from an active, funded follow-up observing program. This program used ground-based radial velocity measurements to determine the mass of exoplanets \citep[e.g.,][]{Marcy2014} when possible and also ruled out other astrophysical phenomenon, like background eclipsing binaries, that can mimic a transit signal.  The follow-up program obtained high-resolution imaging of $\approx$90\% of known KOIs \citep[e.g.,][]{Furlan2017} to identify close companions (bound or unbound) that would be included in \Kepler{'s} rather large 4 arcsecond pixels.  The extra light from these companions must be accounted for when determining the depth of the transit and the radii of the exoplanet.  While the \Kepler{} Pipeline accounts for the stray light from stars in the Kepler Input Catalog (\citealt{Brown2011}; and see flux fraction in \S2.3.1.2 of the Kepler Archive Manual; \citealt{Thompson2014}), the sources identified by these high-resolution imaging catalogs were not included. One study by \citet{Ciardi2015} demonstrated that, on average, planet radii are underestimated by a factor of $\approx$1.5 with the effect being smaller for K and M dwarfs.  
}

%Eclipsing binary systems often mimic the signal of a transit and must be weeded out through either additional confirming data \citep[such as radial velocities, e.g.][]{Marcy2014}, or statistical analysis.

Even with rigorous vetting and follow-up observations, most planet candidates in the KOI catalogs cannot be directly confirmed as planetary. The stars are too dim and the planets are too small to be able to measure a radial velocity signature for the planet.   Statistical methods study the likelihood that the observed transit could be caused by other astrophysical scenarios and have succeeded in validating thousands of \Kepler{} planets \citep[e.g.][]{Lissauer2014,Morton2016,Rowe2014,Torres2015}.  

The Q1--Q16 KOI catalog \citep{Mullally2015cat} was the first with a long enough baseline to be significantly impacted by false positives created by the instrument itself.  In that catalog (and again in this one), the majority of long-period, low SNR TCEs are ascribed to instrumental effects incompletely removed from the data before the TCE search. \Kepler\ has a variety of short timescale (less than half a day), non-Gaussian noise sources including focus changes due to the varying solar illumination angle, signals imprinted on the data by the detector electronics, noise caused by solar flares, and the pixel sensitivity changing after the impact of a high energy particle (known as a sudden pixel sensitivity drop-out, or SPSD). In the development of the vetting methods for the DR25 KOI catalog, we spent considerable effort identifying these types of false positives, and for the first time we include an estimate how these signals contaminate the catalog.

The planet candidates found in \Kepler{} data have been used extensively to understand the frequency of different types of planets in the Galaxy. Many studies have shown that small planets ($<4$\re) in short period orbits are common, with occurrence rates steadily increasing with decreasing radii \citep{Burke2016,Howard2012,Petigura2013b,Youdin2011}.  \citet{Dressing2013,Dressing2015}, using their own search, confined their analysis to M dwarfs and orbital periods less than 50\,d and determined that multi-planet systems are common around low mass stars.  Therefore planets are more common than stars in the Galaxy (due in part to the fact that low mass stars are the most common stellar type). 
\citet{Fulton2017}, using improved measurements of the stellar properties \citep{Petigura2017}, looked at small planets with periods of less than 100\,d and showed that there is a valley in the occurrence of planets near 1.75\re{}. This result improved upon the results of \citet{Howard2012} and \citet{Lundkvist2016} and further verified the evaporation valley predicted by \citet{Owen2013} and \citet{Lopez2013} for close-in planets.

Less is known about the occurrence of planets in longer period orbits. Using planet candidates discovered with \Kepler{}, several papers have measured the frequency of small planets in the habitable zone of sun-like stars \citep[see e.g.][]{Burke2015,ForemanMackey16,Petigura2013b} using various methods. \citet{Burke2015} used the Q1--Q16  KOI catalog \citep{Mullally2015cat} and looked at G and K stars and concluded that 1--200\% of solar-type stars host planets with radii and orbital periods within 20\% of that of the Earth. This range in the planet frequency includes various systematic effects that dominate the uncertainty. A better understanding of the properties of both the stars and the catalog generation process will narrow the uncertainty in such studies. 

 

\subsection{Motivation for the DR25 catalog}

The DR25 KOI catalog is designed to support rigorous occurrence rate studies. To do that well, we not only identify the exoplanet transit signals in the data but also measure the reliability (the fraction of planet candidates that are likely due to astrophysical signals) and the completeness of the catalog (the fraction of true transiting planets detected).

The measurement of the catalog completeness has been split into two parts: the completeness of the TCE list (the transit search performed by the \Kepler\ Pipeline) and the completeness of the KOI catalog (the vetting of the TCEs). The completeness of the \Kepler\ Pipeline and its search for transits has been studied by injecting transit signals into the pixels and examining what fraction are found by the Pipeline \citep{Christiansen2017, Christiansen2015b,Christiansen2013a}. \citet{Burke2015} applied the appropriate detection efficiency contours \citep{Christiansen2015} to the 50--300\,d period planet candidates in the Q1--Q16 KOI catalog \citep{Mullally2015cat} in order to measure the occurrence rates of small planets. However, that study was not able to account for those transit signals correctly identified by the \Kepler{} Pipeline but thrown-out by the vetting process. Along with the DR25 KOI catalog, we provide a measure of the completeness of the DR25 vetting process. 

\Kepler\ light curves contain variability that is not due to planet transits or eclipsing binaries. While the reliability of \Kepler\ catalogs against astrophysical false positives is mostly understood \citep[see e.g.][]{Morton2016}, the reliability against false alarms (a term used in this paper to indicate TCEs caused by intrinsic stellar variability, over-contact binaries, or instrumental noise, i.e., anything that does not look transit-like) has not previously been measured. Instrumental noise, poor detrending, and/or stellar variability can conspire to produce a signal that looks similar to a planet transit. When examining the smallest exoplanets in the longest orbital periods, \citet{Burke2015} demonstrated the importance of understanding the reliability of the catalog, showing that the occurrence of small, earth-like-period planets around G dwarf stars changed by a factor of $\approx$10 depending on the reliability of a few planet candidates.  In this catalog we measure the reliability of the reported planet candidates against this instrumental and stellar noise.  

The completeness of the vetting process is measured by vetting thousands of injected transits found by the \Kepler{} Pipeline. Catalog reliability is measured by vetting signals found in scrambled and inverted \Kepler{} light curves and counting the number of simulated false alarms dispositioned as planet candidates. This desire to vet both the real and simulated TCEs in a reproducible and consistent manner demands an entirely automated method for vetting the TCEs.  

Automated vetting was introduced in the Q1--Q16 KOI catalog \citep{Mullally2015cat} with the Centroid Robovetter and was then extended to all aspects of the vetting process for the DR24 KOI catalog \citep{Coughlin2016}. Because of this automation, the DR24 catalog was the first with a measure of completeness that extended to all parts of the search, from pixels to planets.  Now, with the DR25 KOI catalog and simulated false alarms, we also provide a measure of how effective the vetting techniques are at identifying noise signals and translate that into a measure of the catalog reliability. As a result, the DR25 KOI catalog is the first to balance completeness and reliability, instead of erring on the side of high completeness. 

%%List Acronyms Here
\subsection{Terms and Acronyms}
\label{abbrev}
We try to avoid unnecessary acronyms and abbreviations, but a few are required to efficiently discuss this catalog.  Here we itemize those terms and abbreviations that are specific to this paper and are used repeatedly. The list is short enough that we choose to group them by meaning instead of alphabetically. 

\begin{itemize}
\item[] \textbf{TPS}: Transit Planet Search module. This module of the Kepler pipeline performs the search for planet candidates. This module identifies TCEs.
\item[] \textbf{TCE}: Threshold Crossing Event. Periodic signals identified by the Kepler Pipeline. See Transit Planet Search in \citet{JenkinsKDPH}.
\item[] \textbf{\opstce}: Observed TCEs. TCEs found by searching the observed DR25 \Kepler\ data and reported in \citet{Twicken2016}.
\item[] \textbf{\injtce}: Injected TCE. TCEs found that match a known, injected transit signal \citet{Christiansen2017}.
\item[] \textbf{\invtce}: Inverted TCE. TCEs found when searching the inverted data set in order to simulate instrumental false alarms \citet{Coughlin2017a}.
\item[] \textbf{\scrtce}: Scrambled TCE. TCEs found when searching the scrambled data set in order to simulate instrumental false alarms \citet{Coughlin2017a}.
\item[] \textbf{MES}: Multiple Event Statistic. A statistic that measures the significance of the observed transits in the detrended, whitened light curve \citep{Jenkins2002a}.
\item[] \textbf{KOI}: Kepler Object of Interest. Periodic, transit-like, events that were significant enough to warrant further review. A KOI is identified with a KOI number.
\item[] \textbf{PC}: Planet Candidate. A TCE that passed all of the Robovetter tests and metrics.  
\item[] \textbf{FP}: False Positive. A TCE that failed one or more of the Robovetter tests and metrics.
\item[] \textbf{DV}: Data Validation. Named after the module of the \Kepler\ Pipeline which characterizes the transits and outputs the \Kepler\ Pipeline's light curve detrending. DV is also used to refer to two sets of transit fits: original and supplemental \citet{JenkinsKDPH}.
\item[] \textbf{ALT}: Alternative. An alternative detrending, based on the methods of \citet{Garcia2010}, is utilized by several Robovetter metrics. A trapezoidal fit to the transit is performed on the ALT detrended light curves.
\item[] \textbf{MCMC}: Markov chain Monte Carlo. This refers to the transit fits that are provided for all KOIs \citep{Hoffman2017}.
\item[] \textbf{SPSD}: Sudden Pixel Sensitivity Dropout.  After a cosmic ray hit a pixel can suddenly lose sensitivity and gradually gain sensitivity over a few hours.

\end{itemize}


\subsection{Summary and Outline of the Paper}

The DR25 KOI catalog is a uniformly-vetted list of planet candidates and false positives found by searching the DR25 \Kepler\ light curves, with a measure of the catalog completeness and reliability. In the brief outline that follows we highlight how the catalog was assembled, how we measure the completeness and reliability, and discuss those aspects of the process that are different from the DR24 KOI catalog \citep{Coughlin2016}.

In \S\ref{s:tces} we describe the observed TCEs (\opstce s) which are the periodic signals found in the actual \Kepler\ light curves. For reference, we also compare them to the DR24 TCEs. To create the simulated data sets necessary to measure the vetting completeness and the catalog reliability, we ran the \Kepler\ Pipeline on light curves that either contained injected transits, were inverted, or were scrambled. This creates \injtce s, \invtce s, and \scrtce s, respectively (see \S\ref{s:simulated}).  

We then created and tuned a Robovetter to vet all the different sets of TCEs. \S\ref{s:robovetter} describes the metrics and the logic used to disposition TCEs into planet candidates (PCs) and false positives (FPs).  Because the DR25 \opstce\ population was significantly different than the DR24 \opstce{s}, we developed new metrics to separate the PCs from the FPs (see Appendix \ref{s:metrics} for the details on how each metric operates.) Several new metrics examine the individual transits for evidence of instrumental noise (see \S\ref{s:indivtrans}.) As in the DR24~KOI catalog, we group false positives into four categories (\S\ref{s:majorflags}) and provide false positive flags (Appendix \ref{s:minorflags}) to indicate why the Robovetter decided to pass or fail a TCE.  New to this catalog is the addition of a disposition score (\S\ref{s:scores}) that gives users a measure of the Robovetter's confidence of each disposition. Finally, we fit transit models to each KOI to calculate planet parameters and use a Markov Chain Monte Carlo (MCMC)
algorithm to provide error estimates for each fitted parameter, see \S\ref{s:mcmc}.


Unlike previous catalogs, for the DR25 KOI catalog the choice of planet candidate versus false positive is no longer based on the philosophy of ``innocent until proven guilty''. We accept certain amounts of collateral damage (i.e., exoplanets dispositioned as FP) in order to achieve a catalog that is uniformly vetted and has acceptable levels of both completeness and reliability, especially for the long period and low signal-to-noise PCs. In \S\ref{s:optimize} we discuss how we tuned the Robovetter using the simulated TCEs as populations of true planet candidates and true false alarms. We provide the Robovetter source code and all the Robovetter metrics for all of the sets of TCEs (\opstces, \injtces, \invtces, and \scrtces) to enable users to create a catalog tuned for other regions of parameter space if their scientific goals require it. 

We then proceed to discuss the performance of the vetting and the properties of the catalog. We examine the vetting completeness and catalog reliability using the \injtce, \invtce, and \scrtce\ sets in \S\ref{s:candr}. We show that both decrease significantly with decreasing number of transits and decreasing signal-to-noise.  We then discuss how one may use the disposition scores to identify the highest quality candidates, especially at long periods (\S\ref{s:crscores}.)  We conclude that not all declared planet candidates in our catalog are truly planets, but we can measure what fraction are caused by noise. Because of the interest in terrestrial, temperate planets, we examine the high quality, small candidates in the habitable zone in \S\ref{s:hz}. Finally, in \S\ref{s:occurates}, we give an overview of what must be considered when using this catalog to measure accurate exoplanet occurrence rates, including what information is available in other \Kepler{} products to do this work.



