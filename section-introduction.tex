[THIS SECTION NOT DONE]
In 2004 the \Kepler\ space telescope was launched into an earth trailing orbit with the goal of finding the signal from rocky, habitable transiting exoplanets \citep{Koch2010} around sun-like stars. The telescope observed the brightness of over 150,000 stars for four years looking for a small decrease in the brightness of the star caused by a planet passing between the star and the \Kepler\ spacecraft. 
Since \Kepler's launch in 2009 it has succeeded in identifying X,XXX planet candidates \citep{Borucki2010a,Coughlin2016}. 

X,XXX of these exoplanets have gone on to be ... or validated planets \citep[see most recently][]{Morton2016}.  % \citep[e.g.][]{FOP?}

Because of the \Kepler\ data, the landscape of known exoplanets has expanded, finding smaller planets in more distant orbits from their host stars.  To highlight a few examples,  \Kepler\ found terrestrial planets as small as Kepler-37b \citep{Barclay2013}, a moon sized planet in a 13.3 day period. \Kepler\ has even found small exoplanets in the habitable zone of their host star, e.g. Kepler-186f \citep{Quintana2014} and Kepler-452b \citep{Jenkins2015}.  The Kepler data has also shown how exoplanets are commonly found in exoplanetary systems. These systems can be compact like the system around Kepler-11, where there are six planets all with periods less than XX days. Also, because of the long continuous data set, exoplanets have been found orbiting binary stars, e.g. Kepler-16\citep{Doyle2011}.


The impact of \Kepler\ on our understanding of exoplanets comes not only from these individual systems, but from the plethora of planets of every type that it has found. Understanding how common different types of exoplanets are around different types of stars will be one of its lasting legacy. \Kepler\ has shown us that most stars have exoplanets and that super-Earth and small planets are more common than their larger cousins \citet{Burke2015}.  

And for each of these unique systems, there are a plethora of exoplanets This plethora of planets \Kepler\ has shown that small planets are common 

--Describe KOI Tables, what is a TCE.  Quarters, Seasons.

[NEED MORE History here to set up our work. Volunteers??]

\subsection{Motivation for the DR25 catalog}

The DR25 catalog was developed and designed so that it can be used to measure the occurrence rates of exoplanets. To do that well, we not only needed to find the planet candidates, but we also need to measure the completeness and reliability of those planet candidates.

Measurement of the completeness has been split into two parts: the search for signals and the vetting of those signals. The search for signals is done by the \Kepler\ pipeline which then creates a catalog of TCEs (for the observed data we call these \opstce s).  The completeness of the various parts of the \Kepler\ pipeline has been studied by injecting transit signals into the pixels and examining which are turned into \opstce s \citep{Christiansen2015b,Christiansen2013a}[CHECK THESE REFS]. \citet{Burke2015} applied the resulting completeness contours to the 50--300\,d period planet candidates in the Q1--Q16 KOI catalog \citep{Mullally2015cat} in order to measure the occurrence rates of small planets. However, this study was not able to account for those planets identified by the pipeline but thrown-out by the vetting process. In this paper we only discuss the completeness of the vetting process. To do this we use the TCEs created by pixel-level transit injection (\injtce) and measure the fraction that are identified as planet candidates, a process first done in the DR24 KOI catalog \citet{Coughlin2016}.

When examining the smallest exoplanets in the longest orbital periods, \citet{Burke2015} demonstrated the importance of understanding the reliability of the catalog, showing that the occurence of small temperatue planets around G stars changed by a factor of 10?? depending on the reliability of six?? planet candidates. \Kepler\ light curves contain variability that is not due to planet transits or eclipsing binaries.  Instrumental noise, poor detrending, or stellar variability can conspire to make a signal that looks similar to a planet transit. In this catalog we endeavor to measure the reliability of the candidate catalog against this instrumental and stellar noise (i.e. false alarms).  Note, this is separate from the reliability against astrophysical false positives, such as background eclipsing binaries, which has been well studied by \citet{Morton2016} and \citet{Torres2012} by considering the likelihood that a high signal-to-noise transit was created by a planet orbiting the observed star compared to some other astrophysical scenario.  The reliability we measure here considers whether the signal is actually caused by a periodic transit at all. We do this by searching scrambled and inverted \Kepler light curves for transits in the same way we did the original data and counting the number of ``false" planet candidates we find.

%For this catalog we measure the completeness and the reliability of the vetting process by feeding known transit signals and known false alarms through the exact same process that we feed the signals found by searching the \Kepler\ data.  
This requirement to vet not only the \obstce s, but also the \injtce s, \scrtce s and the \invtce{s} caused us to shift the vetting process from a manual one to one that is entirely automated.  The automation of the vetting was introduced in the Q1--Q16 KOI catalog \citet{Mullally2015cat} with the centroid Robovetter and was then extended to all aspects of the vetting process for the DR24 KOI catalog \citep{Coughlin2016}. With the DR24 Robovetter and \injtce s, the DR24 KOI catalog measured, for first time, the completeness of a KOI catalog, from pixels to planets.  However no simulated false alarms were available and Robovetter tended to err on the side of high completeness.  The DR25 KOI catalog completes the requirements by providing two false alarm TCE lists that allow us to understand how effective the Robovetter is at removing false alarms from the final catalog while also providing injected transits to measure how many planet candidates are lost in the process.  

\subsection{Summary of the Paper}
To create this catalog in a timely manner (Hah!) with the tools at our disposal, we had to make a few decisions that were not necessarily ideal, but yet were sufficient to accomplish our goals.  As a result there are a few confusing pieces to the story of how this catalog as been assembled. To prevent confusion, we place a complete outline of how the catalog was created upfront.

The DR25 KOI catalog brings together the results of running many pieces of software that search for and evaluate potential planet transits. It then uses simulated data sets to measure the completeness and reliability of the resulting exoplanet catalog.  In this quick outline of the paper we attempt to highlight how the catalog was assembled, emphsizing those pieces of the process that were different from the DR24 KOI catalog.

In \S\ref{s:tces} we describe the observed TCEs, \opstce s, which are the signals found by the \Kepler\ pipeline. To create the simulated data sets that we require to measure the completeness and reliability, we also run the \Kepler\ pipeline on light curves which contain injected transits, were inverted, and were scrambled. This creates \injtce s, \invtce s, and \scrtce s, respectively (see \S\ref{s:simulated}.  

We then created and tuned a Robovetter to vet the entire set of TCEs. \S\ref{s:robovetter} describes the metrics and the logic used to disposition TCEs into categories of planet candidates (PCs) or false positives (FPs). Because the DR25 \opstce\ population was significantly different than the DR24 \opstce s, we had to invent new metrics. One type of new metric examines the individual transits for evidence of being due to instrumental noise, see \S\ref{s:indivtrans}.  All of the metrics used by the Robovetter are available at the NASA Exoplanet Archive (NExScI). As in the DR24 KOI catalog we group false positives into four categories and provide false positive flags (\S\ref{s:minorflags} to indicate why the Robovetter decided to pass or fail a TCE.  New to this catalog is the addition of a disposition score, see \S\ref{s:scores}, which gives users a measure of our confidence in the KOI disposition. Finally, we perform a full Markov chain Monte Carlo (MCMC) fit to each KOI, see \S\ref{s:mcmc} in order to provide improved individual planet parameters and reliable error bars.

We fit and report to the KOI table all previously known KOIs that match the ephemeris of an \opstce s. Also, any \opstce s that does not fail the Robovetter due to the not-transit-like flag is made into a KOI, either one that already exists if the ephemerides match, or into a new KOI. New to this catalog, we also include KOIs that marginally fail the Robovetter.   This catalog of KOIs with dispositions, disposition scores and MCMC transit fits is available at NExScI alongside the previous KOI catalogs.

Unlike previous catalog, this DR25 KOI catalog is no longer based on the philosophy of ``innocent until proven guilty"; we accepted certain amounts of collateral damage in order to achieve a catalog that had acceptable levels of both completeness and reliability, especially for the long periods and low signal-to-noise PCs. In \S\ref{s:optimize} we discuss how we tuned the Robovetter using the simulated TCEs as true planet candidates and true false alarms. We provide the Robovetter and all the Robovetter metrics for all of the TCEs (\opstce, \injtce, \invtce, and \scrtce) to enable users to create a catalog tuned for other regions of parameter space if their scientific goals require it. 

We then proceed to analyze the catalog.  We examine the completeness and reliability using the \injtce, \invtce and \scrtce\ sets in \S\ref{s:candr}. These both decrease significantly as the number of transits decreases, the period increases and the significance of the transit drops-off.  We then discuss how one may use the disposition scores to achieve a higher reliability, lower completeness catalog for the weakest, long-period signals in \S\ref{s:crscores}.  We conclude that not all planet candidates in our catalog are truly planet candidates. To emphasize this point and because of the interest in the terrestrial, temperate region of this catalog, we examine those candidates identified in \S\ref{s:hz}. Finally, in \S\ref{s:occrates}, we give an overview of how this catalog can be used to measure occurrence rates.




