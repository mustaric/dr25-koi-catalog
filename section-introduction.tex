In 2004 the \Kepler\ space telescope was launched into an earth trailing orbit with the goal of finding the signal from rocky, habitable transiting exoplanets \citep{Koch2010} around sun-like stars. The telescope observed the brightness of over 150,000 stars for four years looking for a small decrease in the brightness of the star caused by a planet passing between the star and the \Kepler\ spacecraft. 
Since \Kepler's launch in 2009 it has succeeded in identifying X,XXX planet candidates \citep{Borucki2010a,Coughlin2016}. 

X,XXX of these exoplanets have gone on to be ... or validated planets \citep[see most recently][]{Morton2016}.  % \citep[e.g.][]{FOP?}

Because of the \Kepler\ data, the landscape of known exoplanets has expanded, finding smaller planets in more distant orbits from their host stars.  To highlight a few examples,  \Kepler\ found terrestrial planets as small as Kepler-37b \citep{Barclay2013}, a moon sized planet in a 13.3 day period. \Kepler\ has even found small exoplanets in the habitable zone of their host star, e.g. Kepler-186f \citep{Quintana2014} and Kepler-452b \citep{Jenkins2015}.  The Kepler data has also shown how exoplanets are commonly found in exoplanetary systems. These systems can be compact like the system around Kepler-11, where there are six planets all with periods less than XX days. Also, because of the long continuous data set, exoplanets have been found orbiting binary stars, e.g. Kepler-16\citep{Doyle2011}.


The impact of \Kepler\ on our understanding of exoplanets comes not only from these individual systems, but from the plethora of planets of every type that it has found. Understanding how common different types of exoplanets are around different types of stars will be one of its lasting legacy. \Kepler\ has shown us that most stars have exoplanets and that super-Earth and small planets are more common than their larger cousins \citet{Burke2015}.  


And for each of these unique systems, there are a plethora of exoplanets This plethora of planets \Kepler\ has shown that small planets are common 

\begin{enumerate}
\item What is Kepler and what is its mission?
\item How have we made catalogs in the past and why did we switch to a robovetter. What is the robovetter?
\item Introduce what is new in this paper, hint that the population of TCEs is different and that has created challenges in automatic vetting. Introduce how the goal is to get a sense of the completeness and the reliability of the catalog.
\item Introduce the concepts injection and inversion to test the robovetter.
\item Outline how the Robovetter creates Dispositions and the catalog also includes MCMC fits.
\item Create a road map of what is in this paper?
\end{enumerate}

\subsection{Motivation and Summary of the DR25 catalog}

The DR25 catalog was designed so that it can be used to measure the occurrence rates of exoplanets. To do that the catalog needs a measure of the fraction of planets that were found and a measure of the fraction of identified planet candidates that are truly real.  

The measurement of completeness has been split into two parts. The first studies the completeness of the \Kepler\ pipeline, which searches the \Kepler\ light curves and then creates a catalog of TCES (specifically, \opstce s).  The completeness of the various parts of the \Kepler\ pipeline has been studied by \citep{Christiansen2015b,Christiansen2013a}[CHECK THESE REFS]. The final completeness contours were applied to an occurrence rate study by \citep{Burke2015} of the 50--300\,d period planets in the Q1--Q16 KOI catalog \citep{Mullally2015cat}. While this accounts for the majority of the lost planets, it does not account for those lost during the \opstce\ vetting process. This catalog uses transit injection to measure the completeness of the catalog.  

\citet{Burke2015} pointed out the importance of the understanding the reliability of the catalog when measuring the occurrence rate of the smallest planets in the habitable zone.  In this catalog we endeavor to measure the reliability of the candidate catalog against instrumental and stellar noise (false alarms).  Note, this is separate from the reliability against astrophysical false positives, such as background eclipsing binaries, which has been well studied by \citep{Morton2016} and \citet{Torres2012} by considering the likelihood that the observed signal was created by a planet orbiting the observed star and not some other astrophysical scenario.  The reliability we measure here is the possibility that the measured signal is actually caused by a transit or eclipse.

For this catalog we measure the completeness and the reliability of the catalog by feeding known transit signals and known false alarms through the exact same process that we feed the signals found by searching the \Kepler\ data.  This requirement caused us to shift the vetting procedures from a manual process to one that is entirely automated.  This idea was introduced in the Q1--Q16 catalog and then fully implemented in the DR24 KOI Catalog \citep{Coughlin2016}. The DR25 catalog, using this Robovetter, was able to measure the completeness of the catalog using planet injection. This DR25 catalog takes this one step further by measuring both the completeness and the reliability of the DR25 catalog.

\subsection{Summary of the Paper}