In 2004 the \Kepler\ space telescope was launched into an earth trailing orbit with the goal of finding the signal from rocky, habitable transiting exoplanets \citep{Koch2010} around sun-like stars. The telescope observed the brightness of over 150,000 stars for four years looking for a small decrease in the brightness of the star caused by a planet passing between the star and the \Kepler\ spacecraft. 
Since \Kepler's launch in 2009 it has succeeded in identifying X,XXX planet candidates \citep{Borucki2010a,Coughlin2016}. 

X,XXX of these exoplanets have gone on to be ... or validated planets \citep[see most recently][]{Morton2016}.  % \citep[e.g.][]{FOP?}

Because of the \Kepler\ data, the landscape of known exoplanets has expanded, finding smaller planets in more distant orbits from their host stars.  To highlight a few examples,  \Kepler\ found terrestrial planets as small as Kepler-37b \citep{Barclay2013}, a moon sized planet in a 13.3 day period. \Kepler\ has even found small exoplanets in the habitable zone of their host star, e.g. Kepler-186f \citep{Quintana2014} and Kepler-452b \citep{Jenkins2015}.  The Kepler data has also shown how exoplanets are commonly found in exoplanetary systems. These systems can be compact like the system around Kepler-11, where there are six planets all with periods less than XX days. Also, because of the long continuous data set, exoplanets have been found orbiting binary stars, e.g. Kepler-16\citep{Doyle2011}.


The impact of \Kepler\ on our understanding of exoplanets comes not only from these individual systems, but from the plethora of planets of every type that it has found. Understanding how common different types of exoplanets are around different types of stars will be one of its lasting legacy. \Kepler\ has shown us that most stars have exoplanets and that super-Earth and small planets are more common than their larger cousins \citet{Burke2015}.  


And for each of these unique systems, there are a plethora of exoplanets This plethora of planets \Kepler\ has shown that small planets are common 

\begin{enumerate}
\item What is Kepler and what is its mission?
\item How have we made catalogs in the past and why did we switch to a robovetter. What is the robovetter?
\item Introduce what is new in this paper, hint that the population of TCEs is different and that has created challenges in automatic vetting. Introduce how the goal is to get a sense of the completeness and the reliability of the catalog.
\item Introduce the concepts injection and inversion to test the robovetter.
\item Outline how the Robovetter creates Dispositions and the catalog also includes MCMC fits.
\item Create a road map of what is in this paper?
\end{enumerate}
