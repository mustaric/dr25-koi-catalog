\subsection{Multiple-planet systems}
Only 47, or 21 per cent of the new DR25 PCs are associated with targets with multiple-planet candidates. One of the new planet candidates, KOI-82.06, is part of a six planet candidate system around the star Kepler-102. Five candidates have previosly been confirmed \citep{Marcy2014,Rowe2014} in this system. The new candidate is a bit exterior to the 4:3 resonance with the largest verified planet in the system. Also, there is an excess of planets found just wide of such first-order resonances \citep{Lissauer2011}, suggesting that this candidate is likely to be a planet. If verified, this would be only the 3rd system with six or more planets found by Kepler. 
The other new candidate within a high multiplicity system is KOI-2926.05. The other four candidates in this system have been validated as Kepler-1388 by \citet{Morton2016}. This new candidate also orbits just exterior to a first-order mean motion resonance with one of the 4 previously known planets, again adding to the likelihood that it is a true planet.

%This percentage is smaller than the percentage of new planet candidates in the previous catalog that reside in multis.  This decrease is probably related to the increased fraction of long-period candidates among the newly-identified population, as such planets are less likely to be found in multis \citep{Lissauer2014}.  
%[SEM I don't understand what long period means in this context and I don't know whether the previous catalogs had higher than 21% of the new candidates be multis.]